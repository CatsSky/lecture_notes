\lesson{1}{di 13 Feb 2023 12:00}{Introduction}


\begin{itemize}
    \item Score:
    \begin{itemize}
        \item Midterm 30\%
        \item Final 30\%
        \item Assignment 30\%
        \begin{itemize}
            \item 8 questions a week, assigns at every Wednesday.
        \end{itemize}
        \item Attendance 10\%
        \begin{itemize}
            \item About 5 times, 60\% Attendance gets all 10\%
        \end{itemize}
    \end{itemize}
\end{itemize}


\setcounter{chapter}{6}
\chapter{Introduction}

\setcounter{section}{1}
\section{Euler's phi-function}
Goal: Generalize Fermat's little theorem

\begin{theorem}
    Let $p$ be a prime and $a$ an integer.

    If $\gcd(a, p) = 1$, then $a^p-1 \equiv 1 \pmod p$
\end{theorem}

\begin{definition}{}
    For $n \in N$, 
    
    let $\phi(n)$ denotes the number of positive integers not exceeding $n$ that are prime to $n$
\end{definition}
\begin{eg}
    If we put $P_n={1 \leq m \leq n | \gcd(m, n) = 1}$ then $\phi(n) = |P_n|$
\end{eg}

\begin{remark}
    This function is usually called the Euler phi-function.
    Clearly, we have $\phi(1) = 1$
    Suppose $n > 1$ Then since $\gcd(n, 1) = 1$ and $\gcd(n, n) \neq 1$, we find that $1 \leq \phi(n) \leq n-1$
\end{remark}

\begin{eg}
    $P_{30} = {1, 7, 11, 13, 17, 19, 23, 29}$
    $\Rightarrow \phi(30) = |P_{30}| = 8$

    Note that $n$ is a prime $\Leftrightarrow \phi(n) = n-1$
\end{eg}

\setcounter{section}{0}
\section{title}
\begin{theorem}
    If $p$ is a prime and $k > 0$, then
    $\phi(p^k) = p^k - p^{k-1} = p^k (1 - \frac{1}{p})$
\end{theorem}


\begin{explanation}
    let $1 \leq m \leq p^k$ then $\gcd(m, k) = 1 \Leftrightarrow \gcd(m, p) = 1$
    there are $p^{k-1}$ integers between $1$ and $p^k$ divisible by $p$
\end{explanation}

