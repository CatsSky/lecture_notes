\lesson{3}{di 20 Feb 2023 12:00}{}


\begin{theorem}
    Euler's Theorem 

    $a, n \in \Z$ with $n \geq 1$ and $\gcd(a, n) = 1$.

    Then $a^{\phi(n)} \equiv 1 \pmod n$.
\end{theorem}
\begin{proof}
    Another proof of Euler's Theorem.
    Ingredients:
    \begin{itemize}
        \item Fermat's little theorem
        \item Binomial theorem
        \item Formula of $\phi(n)$
    \end{itemize}

    Let $p$ be a prime and $k \in \N$.
    We prove by induction on $k$ that
    \[
        \tag{1}
        a^{\phi(n)} \equiv 1 \pmod n
        \label{eqn:e1}
    \]
    When $k = 1$, $\phi(p^k) = \phi(p) = p - 1$. Then~\eqref{eqn:e1} is
    a content of Fermat's little theorem.

    Assume that~\eqref{eqn:e1} holds for a fixed $k$.

    Then we can write $a^{\phi(p^k)} = 1 + qp^k$ for some $q \in \Z$.
    Since $\phi(p^{k+1}) = p^{k+1} - p^k = p(p^k - p^{k-1}) = p \phi(p^k)$.
    the binomial theorem implies
    \[
        a^{\phi(p^{k+1})} = a^{p\phi(p^k)} = {[a^{\phi(p^k)}]}^p = {(1+qp^k)}^p
        = \sum_{r=0}^{p}{
            \left(
                \begin{array}{c}
                    p \\
                    r
                \end{array}
            \right)
            (qp)
        }
    \]

    Now let $n \in \N$. We may assume $n > 1$.
    Then $n = \prod_{j=1}^{r}{p_{j}^{k_j}}$ where $p_1 < \cdots < p_r$ are
    primes and $k_1, \dots, k_r \in \N$.

    Since $\gcd(a, n) = 1$, we see that $\gcd(a, p_j) = 1$ for $1 \leq j \leq r$.

    Hence by~\eqref{eqn:e1}, we have 
    \[
        a^{\phi(p_i^{k_i})} \equiv 1 \pmod {p_i^{k_i}}
    \]
    for all $1 \leq i \leq r$.
    Since $\phi(n) = \prod_{j=1}^{r}{\phi(p_j^{k_j})}$, we see that
    $\frac{\phi(n)}{\phi(p_i^{k_i})} \in \N$

    Now 
    \[
        \begin{aligned}
            a^{\phi(n)} &= {[a^{\phi(p_i^{k_i})}]}^\frac{\phi(n)}{\phi(p_i^{k_i})} \\
            &\equiv 1^\frac{\phi(n)}{\phi(p_i^{k_i})} \pmod p_i^{k_i} \\
            &\equiv 1 \pmod p_i^{k_i}
        \end{aligned}
    \]
    for $1 \leq i \lq r \Rightarrow p_i^{k_i} | a^{\phi(n)} - 1, \forall i$.

    This implies
    \[
        \begin{aligned}
            n = \prod_{j=1}^{r}{p_j^{k_j}} | a^{\phi(n)} - 1 \\
            \text{i.e. } a^{\phi(n)} \equiv 1 \pmod n
        \end{aligned}
    \]
\end{proof}

\begin{eg}
    Application of Euler's Theorem

    \begin{enumerate}
        \item Let's find the last two digit in the decimal representation of $3^{256}$,
        i.e.\ we want to find $3^{256} \equiv r \pmod {100}$

        Since $\gcd(3, 100) = 1$ and $\phi(100) = \phi(2^2 5^2) = 40$

        Euler's theorem implies 
        \[
            3^{40} \equiv 40 \pmod {100}
        \]
        Since $256 = 40\cdot6+16$, we find that
        \[
            3^{256} = 3^{40\cdot6+16}=3^{16}\cdot{(3^{40})}^6 \equiv 3^{16} \pmod {100}
        \]

        To compute $3^{16}\pmod{100}$, we use the method of succesive squaring:
        \[
            \begin{aligned}
                &3^2 \equiv 9 \pmod{100} \\
                &3^4={(3^2)}^2\equiv9^2\equiv81\pmod{100} \\
                &3^8={(3^4)}^2\equiv81^2\equiv61\pmod{100} \\
                &3^{16}={(3^8)}^2\equiv61^2\equiv21\pmod{100}
            \end{aligned}
        \]
        thus $r = 21$.

        \item Let $n \in \N$ with $\gcd(n, 10) = 1$.
        Then we show that $n$ divides an integer whose digits are all equal to 1.

        (e.g. $n|11111$)

        Since $\gcd(n, 10) = 1$ and $\gcd(9, 10) = 1$,
        we have $\gcd(9n, 10) = 1$, By Euler's Theorem,
        \[
            10^{\phi(9n)} \equiv 1 \pmod{9n}
        \]
        This implies there exists $k \in \N$ s.t.
        \[
            \begin{aligned}
                10^{\phi(9n)} = 1 + 9nk &\Rightarrow nk = \frac{10^{\phi(9n)} - 1}{9} \\
                &\Rightarrow n | \frac{10^{\phi(9n)}-1}{9}
            \end{aligned}
        \]

        \item Recall the Chinese Remainder Theorem:
        
        Let $n_1, \dots, n_r$ be pairwise relatively prime positive integers. 
        Then the linear conguences
        \[
            \tag{*}
            x \equiv a_1 \pmod{n_1}, \dots, x \equiv a_r \pmod{n_r}
            \label{eqn:e2}
        \]
        has a unique solution modulo $n = n_1 n_2 \dots n_r$.
        Here we give another proof by using Euler's Theorem.

        For $1 \leq i \leq r$, let $N_i = \frac{n}{n_i}$. 
        
        Then we have $\gcd(N_i, n_i) = 1$ and 
        $N_i \equiv 0 \pmod{n_j}$ for $1 \leq i \neq j \leq r$.
        Let
        \[
            \begin{aligned}
                a &= \sum_{i=1}^{r}{a_i N_i^{\phi(n_i)}} \\
                &= a_1 N_1^{\phi(n_1)} + \cdots + a_1 N_r^{\phi(n_r)}
            \end{aligned}
        \]

        We claim that $a$ is a solution of~\eqref{eqn:e2}.

        Let $1 \leq j \leq r$. Then since $\gcd(n_j, N_j) = 1$, Euler's Theorem implies
        \[
            N_j^{\phi(n_j)} \equiv 1 \pmod{n_j}
        \]
        On the other hand, for $1 \leq i \neq j \leq r$, we have
        \[
            N_i^{\phi(n_i)} \equiv 0 \pmod{n_j}
        \]

        Therefore,
        \[
            \begin{aligned}
                a = a_j N-J^{\phi(n_j)} + \sum_{i \neq j}^{}{a_i N_i^{\phi(n_i)}} &\equiv a_j \cdot 1 + 0 \pmod{n_j} \\
                &\equiv a_j \pmod n_j
            \end{aligned}
        \]
    \end{enumerate}
\end{eg}

\section{Some Properties of the phi-function}

\begin{theorem}
    7.6 Gauss.

    Let $n \in \Z$. Then $n = \sum_{d | n}^{}\phi(d)$.
\end{theorem}
\begin{proof}
    Let $d \in Z$ e a divisor of $n$. Consider the set
    \[
        S_d = \{ 1 \leq m \leq n | \gcd(m, n) = d \}
    \]
    We claim $|S_d| = \phi(\frac{n}{d})$. For this, recall that
    \[
        \gcd(m, n) = d \iff \gcd(\frac{m}{d}, \frac{n}{d}) = 1
    \]
    Therefore $S_d = \{ dk | 1 \leq k \leq \frac{n}{d}, \gcd(k, \frac{n}{d}) = 1 \}$.\newline
    This implies $|S_d| = \phi(\frac{n}{d})$.

    Next, noting that
    \[
        \{ 1, 2, \dots, n \} = \cup_{d | n} S_d {\And} S_d \cap S_{d\prime} = \O \text{ if } d\neq d\prime
    \]

    Indeed, if $S_d \cap S_{d\prime} \neq \O$, say $m \in S_d \cap S_{d\prime}$, then
    $d = \gcd(m, n) = d\prime$.

    On the other hand, let $m \in \{1, 2, \dots, n\}$ and $d\prime = \gcd(m, n)$.
    Then $m \in S_{d\prime}$ and hence $\{1, 2, \dots, n\} \subseteq \cup_{d | n}S_d$.

    Now implies
    \[
        \begin{aligned}
            &| \{ 1, 2, \dots, n \} | = \sum_{d | n}{ |S_d| } \\
            \Rightarrow &n = \sum_{d|n}^{}{\phi(\frac{n}{d})} = \sum_{d|n}^{}{\phi(d)}
        \end{aligned}
    \]
\end{proof}

\begin{theorem}
    7.7

    For $n > 1$, the sum of the positive integers less than 
    and relatively prime to n is $\frac{1}{2}n\phi(n)$.
\end{theorem}
\begin{proof}
    Let $m = \phi(n)$ and $a_1, a_2, \dots, a_m$ be the integers less than and relitively prime to n.
    Note that
    \[
        \gcd(n, a) = 1 \iff \gcd(n, n-a) = 1
    \]
    Thus we see that $\gcd(n, n-a_j) = 1, \forall 1 \leq j \leq r$.
    Since $1 \leq n-a_j \leq n$ for $1 \leq j \leq m$ and they are all distinct, we find that
    \[
        \{ a_1, \dots, a_m \} = \{ n-a_1, n-a_2, \dots, n-a_m \}
    \]

    Therefore,
    \[
        \begin{aligned}
            2\sum_{i=1}^{m}{a_i} &= \sum_{i=1}^{m}a_i + \sum_{i=1}^{m}a_i \\
            &= \sum_{i=1}^{m}a_i + \sum_{i=1}^{m}{n-a_i} \\
            &= \sum_{i=1}^{m}{a_i + n - a_i} \\
            &= m \cdot n \\
            &= n \cdot \phi(n)
        \end{aligned}
    \]
    And hence $\sum_{i=1}^{m}a_i = \frac{1}{2}n\phi(n)$.
\end{proof}

\begin{note}
    Recall the Mobius $\mu$-function:
    \[
        \mu(n) = \begin{cases}
            1 &\text{if } n = 1\\
            0 &\text{if } p^2 | n \text{ for some prime } p\\
            {(-1)}^r &\text{if } n = p_1 \cdots p_r \text{ where } p_1 < p_2 < \cdots p_r \text{ are primes}\\
        \end{cases}
    \]

    Recall that if $f(n)$ is a number-theoretic function, \newline
    and $F(n) = \sum_{d|n}^{}{f(d)}$,\newline
    then the Mobius inversion formula says
    \[
        f(n) = \sum_{d|n}^{}{\mu(d)F(\frac{n}{d})} = \sum_{d|n}^{}{\mu(\frac{n}{d}F(d))}
    \]
\end{note}

\begin{theorem}
    7.8
    \[
        \phi(n) = n \sum_{d|n}{\frac{\mu(d)}{d}}, \forall n \in \N
    \]
\end{theorem}

\begin{eg}
    Let $n = 10$, Then $\phi(n) = 4$

    On the other hand,
    \[
        \begin{aligned}
            n \sum_{d|n}^{}{\frac{\mu(d)}{d}} &= 10 \sum_{d|10}^{}{\frac{\mu(d)}{d}} \\
            &= 10 [\frac{\mu(1)}{1} + \frac{\mu(2)}{2} + \frac{\mu(5)}{5} + \frac{\mu(10)}{10}] \\
            &= 10 [\frac{1}{1} + \frac{-1}{2} + \frac{-1}{5} + \frac{{(-1)}^2}{10}] \\
            &= 10 \cdot \frac{2}{5} = 4
        \end{aligned}
    \]

    Recall that if $n = \prod_{j=1}^{r}{p^{k_j}_{j}}$ is the prime factorization of $n$,
    then 
    \[
        \tag{*}
        \phi(n) = \prod_{j=1}^{r}{p^{k_j}_{j} - p^{k_j-1}_{j}} = n\prod_{j=1}^{r}{(1-\frac{1}{p_j})}
        \label{eqn:7.4.1}
    \]
    We want to prove~\eqref{eqn:7.4.1} by using Theorem 7.8.
    Since $\phi$ is a multiplicative function, it suffices to show 
    \[
        \tag{**}
        \phi(p^k) = p^k - p^{k-1}
        \label{eqn:7.4.2}
    \]
    where $p$ is a prime and $k \in \N$.

    To prove~\eqref{eqn:7.4.2}, we apply theorem 7.8.
    \[
        \begin{aligned}
            \phi(p^k) &= p^k \sum_{d|p^k}{\frac{\mu(d)}{d}} \\
            &= p^k (\frac{\mu(1)}{1} + \frac{\mu(p)}{p} + \frac{\mu(p^2)}{p^2} + \cdots + \frac{\mu(p^k)}{p^k}) \\
            &= p^k (1 - \frac{1}{p}) \\
            &= p^k - p^{k-1}
        \end{aligned}
    \]
\end{eg}


\chapter{PRIMITIVE ROOTS AND INDICES}

\section{The Order of an Integer Modulo $n$}

\setcounter{definition}{0}
\setcounter{theorem}{0}

\begin{definition}
    Let $n, a$ be integers with $n>1$ and $\gcd(a, n) = 1$.
    The order of $a$ modulo $n$ is the \textbf{smallest} $k \in \N$ s.t. $a^k \equiv 1 \pmod n$.
    We sometimes denote $k$ by $\text{ord}_n(a)$
\end{definition}
\begin{eg}
    Let $n = 7$ and $a = 2$, We compute $\text{ord}_7(2)$:
    \[
        \begin{aligned}
            2^1 \equiv 2 \pmod 7 \\
            2^2 \equiv 4 \pmod 7 \\
            2^3 \equiv 1 \pmod 7 \\
            \Rightarrow \text{ord}_7(2) = 3
        \end{aligned}
    \]

    Note that by Fermat's little theorem: $2^{7-1} \equiv 1 \pmod 7$.

    In general, $2^{3m} \equiv 1 \pmod 7 \; \forall m \in \N$
\end{eg}
\begin{remark}
    .
    \begin{enumerate}
        \item Note that if $a \equiv b \pmod n$, then $\text{ord}_n(a) = \text{ord}_n(b)$\newline
        \emph{Since $a \equiv b \pmod n \Rightarrow a^m \equiv b^m \pmod n , \forall m \in \N$}

        \item If $\gcd(a, n) > 1$, then $a^m \not\equiv 1 \pmod n, \forall m \in \N$\newline
        \emph{If $\gcd(a, n) > 1$ but $a^m \equiv 1 \pmod n$ for some $m \in \N$, then
        the linear conguence $ax \equiv 1 \pmod n$ has a solution $x = a^{m-1}$. This is a countradiction
        since $\gcd(a, b) \not | 1$}
    \end{enumerate}
\end{remark}

\begin{theorem}
    Let $a$ be an integer with order $k$ modulo $n$. Then $a^k \equiv 1 \pmod n$ 
    iff $k | h$. In particular, we have $k | \phi(n)$\newline
    \emph{By Euler's Theorem, $a^{\phi(n)} \equiv 1 \pmod n \Rightarrow k | \phi(n)$}

\end{theorem}
\begin{proof}
    $(\Leftarrow)$ Suppose $k|h$. Then $h = km$ for some $m \in \N$.
    Since $k$ is the order of $a$ modulo $n$, $a^k \equiv 1 \pmod n$.
    Hence $a^h = a^{km} = {(a^k)}^m \equiv 1^m = 1 \pmod n$

    $(\Rightarrow)$ Suppose $a^k \equiv 1 \pmod n$. By division algorithm,
    \[
        h = kq+r
    \]
    for some integer $q, r$ with $0 \leq r < k$. Since $a^k \equiv 1 \pmod n$, 
    we get $1 \equiv a^h = a^{kq+r} = a^r \cdot {(a^k)}^q \equiv a^r \cdot 1^q = a^r \pmod n$.
    Since $r < k$ and $k$ is the order of $a \Rightarrow r = 0 \Rightarrow k | h$
\end{proof}

\begin{remark}
    \begin{enumerate}
        \item Theorem 8.1 implies that in order to find the order of $a$ modulo $n$, it suffices to consider the positive divisors $\phi(n)$
        for example, since $\phi(13) = 12$, Theorem 8.1 implies $\text{ord}_{13}(2 = 1, 2, 3, 4, 6, 12)$. Since
        \[
            \begin{aligned}
                2^1 \equiv 2 \pmod 13 \\
                2^2 \equiv 4 \pmod 13 \\
                2^3 \equiv 8 \pmod 13 \\
                2^4 \equiv 3 \pmod 13 \\
                2^6 \equiv 12 \pmod 13 \\
                2^{12} \equiv 1 \pmod 13 \\
            \end{aligned}
        \] 
        we find $\text{ord}_{13}(2) = 12$

        \item Conversely, for an arbitrary selected divisor $d$ of $\phi(n)$, it's not always true
        that there exists an integer having order $d$ modulo $n$. For example, take $n = 12$.
        Then $\phi(12) = 4$, yet there is no integer that is of order $4$ modulo $12$.
        \[
            \begin{aligned}
                &1 \equiv 5^2 \equiv 7^2 \equiv 11^2 \pmod{12} \\
                &\Rightarrow \text{ord}_{12}(1) = 1, \text{ord}_{12}(5) = \text{ord}_{12}(7) = \text{ord}_{12}(11) = 2
            \end{aligned}
        \]
    \end{enumerate}
\end{remark}

\begin{theorem}
    If $\text{ord}_{n}(a) = k$, then $a^i \equiv a^j \pmod n$ iff $k | i - j$, i.e. $i \equiv j \pmod k$.
\end{theorem}
\begin{proof}
    $(\Rightarrow)$ Suppose $a^i \equiv a^j \pmod n$ and $i \geq j$.
    Then $n | a^j - a^i = a^j(a^{i-j} - 1)$.
    Since $\gcd(a, n) = 1$, we find that $n | a^{i-j} - 1$, i.e.
    \[
        a^{i-j} \equiv 1 \pmod n
    \]
    Now theorem 8.1 implies, $k | i-j$, i.e. $i \equiv j \pmod k$

    $(\Leftarrow)$ Suppose $i \equiv j \pmod n$ and $i \geq j$ and 
    Theorem 8.1 says $a^{i-j} \equiv 1 \pmod n \Rightarrow a^{(i-j)} a^j \equiv a^j \pmod n$
\end{proof}
\begin{corollary}
    If $a$ has order $k$ modulo $n$, then the integers $a, a^2, \dots, a^k$ are conguent modulo $n$.
\end{corollary}
\begin{proof}
    If $a^i \equiv a^j \pmod n$ for some $1 \leq i, j \leq k$, then theorem 8.2 implies $i \equiv j \pmod k$.
    Since $1 \leq i, j \leq k$, the condition $i \equiv j \pmod k$ gives $i = j$
\end{proof}

\begin{theorem}
    If an integer $a$ has the order $k$ modulo $n$ and $h > 0$. 
    Then $a^h$ has an order $\frac{k}{\gcd(k, h)}$ modulo $n$
\end{theorem}
\begin{proof}
    Let $d = \gcd(k, h)$ and $k = dk\prime$, $h = dh\prime$, 
    for some $k\prime, h\prime \in \N$ with $\gcd(k\prime, h\prime) = 1$.
    Then we have $\frac{k}{\gcd(k, h)} = \frac{k}{d} = k\prime$ and
    \[
        \begin{aligned}
            {(a^h)}^{\frac{k}{\gcd(k, h)}} &= {(adh\prime)}^{k\prime} \\
            &= a^{dh\prime k\prime} \\
            &= a^{kh\prime} \\
            &= {(a^k)}^{h\prime} \equiv 1 \pmod n
        \end{aligned}
    \]

    By theorem 8.1, we have $r | k\prime$ where $r = \text{ord}_N(a^h)$.
    On the other hand, since $a$ has order $k$ modulo $n$ and $1 \equiv {(a^h)}^r = a^{hr} \pmod n$,
    we get $k | hr$ by theorem 8.1 again. This implies $k\prime | h\prime r$. Since
    $\gcd(k\prime, h\prime) = 1$, we find that $k\prime | r$. Thus $r = k\prime = \frac{k}{\gcd(k, h)}$.
\end{proof}

% skipped some 

\begin{theorem}
    $\text{ord}_{n}(a) | \phi(n) \Rightarrow \text{ord}_{n}(a) \leq \phi(n)$ 
\end{theorem}
\begin{proof}
    Let $a, n \in \Z$ with $n > 1$ and $\gcd(a, n) = 1$. If the order of $a$ mudulo $n$ is $\phi(n)$, i.e.
    $\text{ord}_{n}(a) = \phi(n)$.
    % not finished
\end{proof}


\begin{remark}
    \textbf{NOT} every $n$ has a primitive root!
\end{remark}
\begin{eg}
    \[
        \begin{aligned}        
            & n=12, a=1, 5, 7, 11 \\
            & \phi(12) = 4 \\
            & 1 = 1 \pmod {12} \Rightarrow \text{ord}_{12}(1) = 1 \\
            & 5^2 \equiv 7^2 \equiv 11^2 \equiv 1 \pmod {12} \\
            & \Rightarrow \text{ord}_{12}(5) = \text{ord}_{12}(7) = \text{ord}_{12}(11) = 2 \neq 4
        \end{aligned}
    \]
\end{eg}
\begin{remark}
    Primitive root may not be unique.
\end{remark}


\begin{theorem}
    Let $\gcd(a, n) = 1$ and $a_1, \dots, 1_{\phi(n)}$ be the positive
    integers less than and relatively prime to $n$.

    If $a$ is a primitive root of $n$, then $a, a^2, \dots, a^{\phi(n)}$
    are conguent mudulo $n$ to $a_1, \dots, 1_{\phi(n)}$ is same order.
\end{theorem}
\begin{proof}
    Since $\gcd(a, n) = 1$, we have $\gcd(a^i, n) = 1$ for all $i \in \N$.
    It follows that $a^i \equiv a_{k_i} \pmod n$ for some $1 \leq k_i \leq \phi(n)$.

    By Cor.\ to Theorem 8.2, the $\phi(n)$ integers in $\{a, a^2, \dots, a^{\phi(n)}\}$
    are conguent modulo n, i.e. $a^i \equiv a^j \pmod n, \; \forall 1 \leq i \neq j \leq \phi(n)$.
    $\Rightarrow a_{k_i} \not \equiv a_{k_j} \pmod n,\; \forall 1 \leq i \neq j \leq \phi(n)$.

    Therefore, we get $\{a, a^2, \dots, a^{\phi()}\} \pmod n = \{a_1, \dots, a_{\phi(n)}\}$.
\end{proof}

\begin{corollary}
    If $n$ has a primitive root, then there are exact $\phi(\phi(n))$ of them.
\end{corollary}
\begin{proof}
    Suppose that $a$ is a primitive root of $n$.

    By Theorem 8.4, any other primitive root mudulo $n$ is found in the members of the set 
    $\{a, a^2, \dots, a^{\phi(n)}\}$.

     We know that $\text{ord}_n(a^h) = \frac{\phi(n)}{\gcd(h, \phi(n))}$ for $1 \leq h \leq \phi(n)$
     $\Rightarrow \text{ord}_n(a^h) = \phi(n) \iff \gcd(h, \phi(n)) = 1$
    there are $\phi(\phi(n))$ such $h$.
\end{proof}
\begin{eg}
    \[
        n=13, a=2, \phi(n) = 12
    \]
    \[
        \{2^1, 2^2, \dots, 2^{12}\}
    \]
    \[
        1 \leq k \leq 12, \gcd(12, k) = 1 \Rightarrow 1, 55, 7, 11 \Rightarrow 2^1, 2^5, 2^7, 2^{11} \pmod {13}
    \]
\end{eg}