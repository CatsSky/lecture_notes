\lesson{3}{di 20 Feb 2023 12:00}{}


\begin{theorem}
    Euler's Theorem 

    $a, n \in \Z$ with $n \geq 1$ and $\gcd(a, n) = 1$.

    Then $a^{\phi(n)} \equiv 1 \pmod n$.
\end{theorem}
\begin{proof}
    Another proof of Euler's Theorem.
    Ingredients:
    \begin{itemize}
        \item Fermat's little theorem
        \item Binomial theorem
        \item Formula of $\phi(n)$
    \end{itemize}

    Let $p$ be a prime and $k \in \N$.
    We prove by induction on $k$ that
    \[
        \tag{1}
        a^{\phi(n)} \equiv 1 \pmod n
        \label{eqn:e1}
    \]
    When $k = 1$, $\phi(p^k) = \phi(p) = p - 1$. Then~\eqref{eqn:e1} is
    a content of Fermat's little theorem.

    Assume that~\eqref{eqn:e1} holds for a fixed $k$.

    Then we can write $a^{\phi(p^k)} = 1 + qp^k$ for some $q \in \Z$.
    Since $\phi(p^{k+1}) = p^{k+1} - p^k = p(p^k - p^{k-1}) = p \phi(p^k)$.
    the binomial theorem implies
    \[
        a^{\phi(p^{k+1})} = a^{p\phi(p^k)} = {[a^{\phi(p^k)}]}^p = {(1+qp^k)}^p
        = \sum_{r=0}^{p}{
            \left(
                \begin{array}{c}
                    p \\
                    r
                \end{array}
            \right)
            (qp)
        }
    \]

    Now let $n \in \N$. We may assume $n > 1$.
    Then $n = \prod_{j=1}^{r}{p_{j}^{k_j}}$ where $p_1 < \cdots < p_r$ are
    primes and $k_1, \dots, k_r \in \N$.

    Since $\gcd(a, n) = 1$, we see that $\gcd(a, p_j) = 1$ for $1 \leq j \leq r$.

    Hence by~\eqref{eqn:e1}, we have 
    \[
        a^{\phi(p_i^{k_i})} \equiv 1 \pmod {p_i^{k_i}}
    \]
    for all $1 \leq i \leq r$.
    Since $\phi(n) = \prod_{j=1}^{r}{\phi(p_j^{k_j})}$, we see that
    $\frac{\phi(n)}{\phi(p_i^{k_i})} \in \N$

    Now 
    \[
        \begin{aligned}
            a^{\phi(n)} &= {[a^{\phi(p_i^{k_i})}]}^\frac{\phi(n)}{\phi(p_i^{k_i})} \\
            &\equiv 1^\frac{\phi(n)}{\phi(p_i^{k_i})} \pmod p_i^{k_i} \\
            &\equiv 1 \pmod p_i^{k_i}
        \end{aligned}
    \]
    for $1 \leq i \lq r \Rightarrow p_i^{k_i} | a^{\phi(n)} - 1, \forall i$.

    This implies
    \[
        \begin{aligned}
            n = \prod_{j=1}^{r}{p_j^{k_j}} | a^{\phi(n)} - 1 \\
            \text{i.e. } a^{\phi(n)} \equiv 1 \pmod n
        \end{aligned}
    \]
\end{proof}

\begin{eg}
    Application of Euler's Theorem

    \begin{enumerate}
        \item Let's find the last two digit in the decimal representation of $3^{256}$,
        i.e.\ we want to find $3^{256} \equiv r \pmod {100}$

        Since $\gcd(3, 100) = 1$ and $\phi(100) = \phi(2^2 5^2) = 40$

        Euler's theorem implies 
        \[
            3^{40} \equiv 40 \pmod {100}
        \]
        Since $256 = 40\cdot6+16$, we find that
        \[
            3^{256} = 3^{40\cdot6+16}=3^{16}\cdot{(3^{40})}^6 \equiv 3^{16} \pmod {100}
        \]

        To compute $3^{16}\pmod{100}$, we use the method of succesive squaring:
        \[
            \begin{aligned}
                &3^2 \equiv 9 \pmod{100} \\
                &3^4={(3^2)}^2\equiv9^2\equiv81\pmod{100} \\
                &3^8={(3^4)}^2\equiv81^2\equiv61\pmod{100} \\
                &3^{16}={(3^8)}^2\equiv61^2\equiv21\pmod{100}
            \end{aligned}
        \]
        thus $r = 21$.

        \item Let $n \in \N$ with $\gcd(n, 10) = 1$.
        Then we show that $n$ divides an integer whose digits are all equal to 1.

        (e.g. $n|11111$)

        Since $\gcd(n, 10) = 1$ and $\gcd(9, 10) = 1$,
        we have $\gcd(9n, 10) = 1$, By Euler's Theorem,
        \[
            10^{\phi(9n)} \equiv 1 \pmod{9n}
        \]
        This implies there exists $k \in \N$ s.t.
        \[
            \begin{aligned}
                10^{\phi(9n)} = 1 + 9nk &\Rightarrow nk = \frac{10^{\phi(9n)} - 1}{9} \\
                &\Rightarrow n | \frac{10^{\phi(9n)}-1}{9}
            \end{aligned}
        \]

        \item Recall the Chinese Remainder Theorem:
        
        Let $n_1, \dots, n_r$ be pairwise relatively prime positive integers. 
        Then the linear congruences
        \[
            \tag{*}
            x \equiv a_1 \pmod{n_1}, \dots, x \equiv a_r \pmod{n_r}
            \label{eqn:e2}
        \]
        has a unique solution modulo $n = n_1 n_2 \dots n_r$.
        Here we give another proof by using Euler's Theorem.

        For $1 \leq i \leq r$, let $N_i = \frac{n}{n_i}$. 
        
        Then we have $\gcd(N_i, n_i) = 1$ and 
        $N_i \equiv 0 \pmod{n_j}$ for $1 \leq i \neq j \leq r$.
        Let
        \[
            \begin{aligned}
                a &= \sum_{i=1}^{r}{a_i N_i^{\phi(n_i)}} \\
                &= a_1 N_1^{\phi(n_1)} + \cdots + a_1 N_r^{\phi(n_r)}
            \end{aligned}
        \]

        We claim that $a$ is a solution of~\eqref{eqn:e2}.

        Let $1 \leq j \leq r$. Then since $\gcd(n_j, N_j) = 1$, Euler's Theorem implies
        \[
            N_j^{\phi(n_j)} \equiv 1 \pmod{n_j}
        \]
        On the other hand, for $1 \leq i \neq j \leq r$, we have
        \[
            N_i^{\phi(n_i)} \equiv 0 \pmod{n_j}
        \]

        Therefore,
        \[
            \begin{aligned}
                a = a_j N-J^{\phi(n_j)} + \sum_{i \neq j}^{}{a_i N_i^{\phi(n_i)}} &\equiv a_j \cdot 1 + 0 \pmod{n_j} \\
                &\equiv a_j \pmod n_j
            \end{aligned}
        \]
    \end{enumerate}
\end{eg}

\section{Some Properties of the phi-function}

\begin{theorem}
    7.6 Gauss.

    Let $n \in \Z$. Then $n = \sum_{d | n}^{}\phi(d)$.
\end{theorem}
\begin{proof}
    Let $d \in Z$ e a divisor of $n$. Consider the set
    \[
        S_d = \{ 1 \leq m \leq n | \gcd(m, n) = d \}
    \]
    We claim $|S_d| = \phi(\frac{n}{d})$. For this, recall that
    \[
        \gcd(m, n) = d \iff \gcd(\frac{m}{d}, \frac{n}{d}) = 1
    \]
    Therefore $S_d = \{ dk | 1 \leq k \leq \frac{n}{d}, \gcd(k, \frac{n}{d}) = 1 \}$.\newline
    This implies $|S_d| = \phi(\frac{n}{d})$.

    Next, noting that
    \[
        \{ 1, 2, \dots, n \} = \cup_{d | n} S_d {\And} S_d \cap S_{d\prime} = \O \text{ if } d\neq d\prime
    \]

    Indeed, if $S_d \cap S_{d\prime} \neq \O$, say $m \in S_d \cap S_{d\prime}$, then
    $d = \gcd(m, n) = d\prime$.

    On the other hand, let $m \in \{1, 2, \dots, n\}$ and $d\prime = \gcd(m, n)$.
    Then $m \in S_{d\prime}$ and hence $\{1, 2, \dots, n\} \subseteq \cup_{d | n}S_d$.

    Now implies
    \[
        \begin{aligned}
            &| \{ 1, 2, \dots, n \} | = \sum_{d \divides n}{ |S_d| } \\
            \Rightarrow &n = \sum_{d \divides n}^{}{\phi(\frac{n}{d})} = \sum_{d|n}^{}{\phi(d)}
        \end{aligned}
    \]
\end{proof}

\begin{theorem}
    7.7

    For $n > 1$, the sum of the positive integers less than 
    and relatively prime to n is $\frac{1}{2}n\phi(n)$.
\end{theorem}
\begin{proof}
    Let $m = \phi(n)$ and $a_1, a_2, \dots, a_m$ be the integers less than and relitively prime to n.
    Note that
    \[
        \gcd(n, a) = 1 \iff \gcd(n, n-a) = 1
    \]
    Thus we see that $\gcd(n, n-a_j) = 1, \forall 1 \leq j \leq r$.
    Since $1 \leq n-a_j \leq n$ for $1 \leq j \leq m$ and they are all distinct, we find that
    \[
        \{ a_1, \dots, a_m \} = \{ n-a_1, n-a_2, \dots, n-a_m \}
    \]

    Therefore,
    \[
        \begin{aligned}
            2\sum_{i=1}^{m}{a_i} &= \sum_{i=1}^{m}a_i + \sum_{i=1}^{m}a_i \\
            &= \sum_{i=1}^{m}a_i + \sum_{i=1}^{m}{n-a_i} \\
            &= \sum_{i=1}^{m}{a_i + n - a_i} \\
            &= m \cdot n \\
            &= n \cdot \phi(n)
        \end{aligned}
    \]
    And hence $\sum_{i=1}^{m}a_i = \frac{1}{2}n\phi(n)$.
\end{proof}

\begin{note}
    Recall the Mobius $\mu$-function:
    \[
        \mu(n) = \begin{cases}
            1 &\text{if } n = 1\\
            0 &\text{if } p^2 \divides n \text{ for some prime } p\\
            {(-1)}^r &\text{if } n = p_1 \cdots p_r \text{ where } p_1 < p_2 < \cdots p_r \text{ are primes}\\
        \end{cases}
    \]

    Recall that if $f(n)$ is a number-theoretic function, \newline
    and $F(n) = \sum_{d \divides n}^{}{f(d)}$,\newline
    then the Mobius inversion formula says
    \[
        f(n) = \sum_{d \divides n}^{}{\mu(d)F(\frac{n}{d})} = \sum_{d \divides n}^{}{\mu(\frac{n}{d}F(d))}
    \]
\end{note}

\begin{theorem}
    7.8
    \[
        \phi(n) = n \sum_{d|n}{\frac{\mu(d)}{d}}, \forall n \in \N
    \]
\end{theorem}

\begin{eg}
    Let $n = 10$, Then $\phi(n) = 4$

    On the other hand,
    \[
        \begin{aligned}
            n \sum_{d|n}^{}{\frac{\mu(d)}{d}} &= 10 \sum_{d|10}^{}{\frac{\mu(d)}{d}} \\
            &= 10 [\frac{\mu(1)}{1} + \frac{\mu(2)}{2} + \frac{\mu(5)}{5} + \frac{\mu(10)}{10}] \\
            &= 10 [\frac{1}{1} + \frac{-1}{2} + \frac{-1}{5} + \frac{{(-1)}^2}{10}] \\
            &= 10 \cdot \frac{2}{5} = 4
        \end{aligned}
    \]

    Recall that if $n = \prod_{j=1}^{r}{p^{k_j}_{j}}$ is the prime factorization of $n$,
    then 
    \[
        \tag{*}
        \phi(n) = \prod_{j=1}^{r}{p^{k_j}_{j} - p^{k_j-1}_{j}} = n\prod_{j=1}^{r}{(1-\frac{1}{p_j})}
        \label{eqn:7.4.1}
    \]
    We want to prove~\eqref{eqn:7.4.1} by using Theorem 7.8.
    Since $\phi$ is a multiplicative function, it suffices to show 
    \[
        \tag{**}
        \phi(p^k) = p^k - p^{k-1}
        \label{eqn:7.4.2}
    \]
    where $p$ is a prime and $k \in \N$.

    To prove~\eqref{eqn:7.4.2}, we apply theorem 7.8.
    \[
        \begin{aligned}
            \phi(p^k) &= p^k \sum_{d|p^k}{\frac{\mu(d)}{d}} \\
            &= p^k (\frac{\mu(1)}{1} + \frac{\mu(p)}{p} + \frac{\mu(p^2)}{p^2} + \cdots + \frac{\mu(p^k)}{p^k}) \\
            &= p^k (1 - \frac{1}{p}) \\
            &= p^k - p^{k-1}
        \end{aligned}
    \]
\end{eg}


\chapter{PRIMITIVE ROOTS AND INDICES}

\section{The Order of an Integer Modulo $n$}

\setcounter{definition}{0}
\setcounter{theorem}{0}

\begin{definition}
    Let $n, a$ be integers with $n>1$ and $\gcd(a, n) = 1$.
    The order of $a$ modulo $n$ is the \textbf{smallest} $k \in \N$ s.t. $a^k \equiv 1 \pmod n$.
    We sometimes denote $k$ by $\text{ord}_n(a)$
\end{definition}
\begin{eg}
    Let $n = 7$ and $a = 2$, We compute $\text{ord}_7(2)$:
    \[
        \begin{aligned}
            2^1 \equiv 2 \pmod 7 \\
            2^2 \equiv 4 \pmod 7 \\
            2^3 \equiv 1 \pmod 7 \\
            \Rightarrow \text{ord}_7(2) = 3
        \end{aligned}
    \]

    Note that by Fermat's little theorem: $2^{7-1} \equiv 1 \pmod 7$.

    In general, $2^{3m} \equiv 1 \pmod 7 \; \forall m \in \N$
\end{eg}
\begin{remark}
    \begin{enumerate}
        \item Note that if $a \equiv b \pmod n$, then $\text{ord}_n(a) = \text{ord}_n(b)$\newline
        \emph{Since $a \equiv b \pmod n \Rightarrow a^m \equiv b^m \pmod n , \forall m \in \N$}

        \item If $\gcd(a, n) > 1$, then $a^m \not\equiv 1 \pmod n, \forall m \in \N$\newline
        \emph{If $\gcd(a, n) > 1$ but $a^m \equiv 1 \pmod n$ for some $m \in \N$, then
        the linear congruence $ax \equiv 1 \pmod n$ has a solution $x = a^{m-1}$. This is a countradiction
        since $\gcd(a, b) \not \divides 1$}
    \end{enumerate}
\end{remark}

\begin{theorem}
    Let $a$ be an integer with order $k$ modulo $n$. Then $a^k \equiv 1 \pmod n$ 
    iff $k | h$. In particular, we have $k | \phi(n)$\newline
    \emph{By Euler's Theorem, $a^{\phi(n)} \equiv 1 \pmod n \Rightarrow k | \phi(n)$}

\end{theorem}
\begin{proof}
    $(\Leftarrow)$ Suppose $k|h$. Then $h = km$ for some $m \in \N$.
    Since $k$ is the order of $a$ modulo $n$, $a^k \equiv 1 \pmod n$.
    Hence $a^h = a^{km} = {(a^k)}^m \equiv 1^m = 1 \pmod n$

    $(\Rightarrow)$ Suppose $a^k \equiv 1 \pmod n$. By division algorithm,
    \[
        h = kq+r
    \]
    for some integer $q, r$ with $0 \leq r < k$. Since $a^k \equiv 1 \pmod n$, 
    we get $1 \equiv a^h = a^{kq+r} = a^r \cdot {(a^k)}^q \equiv a^r \cdot 1^q = a^r \pmod n$.
    Since $r < k$ and $k$ is the order of $a \Rightarrow r = 0 \Rightarrow k | h$
\end{proof}

\begin{remark}
    \begin{enumerate}
        \item Theorem 8.1 implies that in order to find the order of $a$ modulo $n$, it suffices to consider the positive divisors $\phi(n)$
        for example, since $\phi(13) = 12$, Theorem 8.1 implies $\text{ord}_{13}(2 = 1, 2, 3, 4, 6, 12)$. Since
        \[
            \begin{aligned}
                2^1 \equiv 2 \pmod {13} \\
                2^2 \equiv 4 \pmod {13} \\
                2^3 \equiv 8 \pmod {13} \\
                2^4 \equiv 3 \pmod {13} \\
                2^6 \equiv 12 \pmod {13} \\
                2^{12} \equiv 1 \pmod {13} \\
            \end{aligned}
        \] 
        we find $\text{ord}_{13}(2) = 12$

        \item Conversely, for an arbitrary selected divisor $d$ of $\phi(n)$, it's not always true
        that there exists an integer having order $d$ modulo $n$. For example, take $n = 12$.
        Then $\phi(12) = 4$, yet there is no integer that is of order $4$ modulo $12$.
        \[
            \begin{aligned}
                &1 \equiv 5^2 \equiv 7^2 \equiv 11^2 \pmod{12} \\
                &\Rightarrow \text{ord}_{12}(1) = 1, \text{ord}_{12}(5) = \text{ord}_{12}(7) = \text{ord}_{12}(11) = 2
            \end{aligned}
        \]
    \end{enumerate}
\end{remark}

\begin{theorem}
    If $\text{ord}_{n}(a) = k$, then $a^i \equiv a^j \pmod n$ iff $k | i - j$, i.e. $i \equiv j \pmod k$.
\end{theorem}
\begin{proof}
    $(\Rightarrow)$ Suppose $a^i \equiv a^j \pmod n$ and $i \geq j$.
    Then $n | a^j - a^i = a^j(a^{i-j} - 1)$.
    Since $\gcd(a, n) = 1$, we find that $n | a^{i-j} - 1$, i.e.
    \[
        a^{i-j} \equiv 1 \pmod n
    \]
    Now theorem 8.1 implies, $k | i-j$, i.e. $i \equiv j \pmod k$

    $(\Leftarrow)$ Suppose $i \equiv j \pmod n$ and $i \geq j$ and 
    Theorem 8.1 says $a^{i-j} \equiv 1 \pmod n \Rightarrow a^{(i-j)} a^j \equiv a^j \pmod n$
\end{proof}
\begin{corollary}
    If $a$ has order $k$ modulo $n$, then the integers $a, a^2, \dots, a^k$ are congruent modulo $n$.
\end{corollary}
\begin{proof}
    If $a^i \equiv a^j \pmod n$ for some $1 \leq i, j \leq k$, then theorem 8.2 implies $i \equiv j \pmod k$.
    Since $1 \leq i, j \leq k$, the condition $i \equiv j \pmod k$ gives $i = j$
\end{proof}

\begin{theorem}
    If an integer $a$ has the order $k$ modulo $n$ and $h > 0$. 
    Then $a^h$ has an order $\frac{k}{\gcd(k, h)}$ modulo $n$
\end{theorem}
\begin{proof}
    Let $d = \gcd(k, h)$ and $k = dk\prime$, $h = dh\prime$, 
    for some $k\prime, h\prime \in \N$ with $\gcd(k\prime, h\prime) = 1$.
    Then we have $\frac{k}{\gcd(k, h)} = \frac{k}{d} = k\prime$ and
    \[
        \begin{aligned}
            {(a^h)}^{\frac{k}{\gcd(k, h)}} &= {(adh\prime)}^{k\prime} \\
            &= a^{dh\prime k\prime} \\
            &= a^{kh\prime} \\
            &= {(a^k)}^{h\prime} \equiv 1 \pmod n
        \end{aligned}
    \]

    By theorem 8.1, we have $r | k\prime$ where $r = \text{ord}_N(a^h)$.
    On the other hand, since $a$ has order $k$ modulo $n$ and $1 \equiv {(a^h)}^r = a^{hr} \pmod n$,
    we get $k | hr$ by theorem 8.1 again. This implies $k\prime | h\prime r$. Since
    $\gcd(k\prime, h\prime) = 1$, we find that $k\prime | r$. Thus $r = k\prime = \frac{k}{\gcd(k, h)}$.
\end{proof}

% skipped some 

\begin{theorem}
    $\text{ord}_{n}(a) | \phi(n) \Rightarrow \text{ord}_{n}(a) \leq \phi(n)$ 
\end{theorem}
\begin{proof}
    Let $a, n \in \Z$ with $n > 1$ and $\gcd(a, n) = 1$. If the order of $a$ mudulo $n$ is $\phi(n)$, i.e.
    $\text{ord}_{n}(a) = \phi(n)$.
    % not finished
\end{proof}


\begin{remark}
    \textbf{NOT} every $n$ has a primitive root!
\end{remark}
\begin{eg}
    \[
        \begin{aligned}        
            & n=12, a=1, 5, 7, 11 \\
            & \phi(12) = 4 \\
            & 1 = 1 \pmod {12} \Rightarrow \text{ord}_{12}(1) = 1 \\
            & 5^2 \equiv 7^2 \equiv 11^2 \equiv 1 \pmod {12} \\
            & \Rightarrow \text{ord}_{12}(5) = \text{ord}_{12}(7) = \text{ord}_{12}(11) = 2 \neq 4
        \end{aligned}
    \]
\end{eg}
\begin{remark}
    Primitive root may not be unique.
\end{remark}


\begin{theorem}
    Let $\gcd(a, n) = 1$ and $a_1, \dots, 1_{\phi(n)}$ be the positive
    integers less than and relatively prime to $n$.

    If $a$ is a primitive root of $n$, then $a, a^2, \dots, a^{\phi(n)}$
    are congruent mudulo $n$ to $a_1, \dots, 1_{\phi(n)}$ is same order.
\end{theorem}
\begin{proof}
    Since $\gcd(a, n) = 1$, we have $\gcd(a^i, n) = 1$ for all $i \in \N$.
    It follows that $a^i \equiv a_{k_i} \pmod n$ for some $1 \leq k_i \leq \phi(n)$.

    By Cor.\ to Theorem 8.2, the $\phi(n)$ integers in $\{a, a^2, \dots, a^{\phi(n)}\}$
    are congruent modulo n, i.e. $a^i \equiv a^j \pmod n, \; \forall 1 \leq i \neq j \leq \phi(n)$.
    $\Rightarrow a_{k_i} \not \equiv a_{k_j} \pmod n,\; \forall 1 \leq i \neq j \leq \phi(n)$.

    Therefore, we get $\{a, a^2, \dots, a^{\phi()}\} \pmod n = \{a_1, \dots, a_{\phi(n)}\}$.
\end{proof}

\begin{corollary}
    If $n$ has a primitive root, then there are exact $\phi(\phi(n))$ of them.
\end{corollary}
\begin{proof}
    Suppose that $a$ is a primitive root of $n$.

    By Theorem 8.4, any other primitive root mudulo $n$ is found in the members of the set 
    $\{a, a^2, \dots, a^{\phi(n)}\}$.

     We know that $\text{ord}_n(a^h) = \frac{\phi(n)}{\gcd(h, \phi(n))}$ for $1 \leq h \leq \phi(n)$
     $\Rightarrow \text{ord}_n(a^h) = \phi(n) \iff \gcd(h, \phi(n)) = 1$
    there are $\phi(\phi(n))$ such $h$.
\end{proof}
\begin{eg}
    \[
        n=13, a=2, \phi(n) = 12
    \]
    \[
        \{2^1, 2^2, \dots, 2^{12}\}
    \]
    \[
        1 \leq k \leq 12, \gcd(12, k) = 1 \Rightarrow 1, 55, 7, 11 \Rightarrow 2^1, 2^5, 2^7, 2^{11} \pmod {13}
    \]
\end{eg}

% extra

\subsection*{RSA Encryption}

Let $p, q$ be distinct primes.
Let $n = p \cdot q \Rightarrow \phi(n) = \phi(p - 1)\phi(q - 1) = (p - 1)(q - 1)$.\newline
Pick $a, b \in \Z$ satisfy $a \cdot b \equiv 1 \pmod{\phi(n)}$.

\begin{itemize}
    \item Encryption\newline
    $x \in \Z_n \Rightarrow e_b(x) = x^b \pmod n$
    
    \item Decryption\newline
    $y \in \Z_n \Rightarrow d_a(y) = y^a \pmod n$

    \item Check\newline
    $d_a(e_b(x)) = x \Rightarrow x^{ab} \equiv x \pmod n, \;\forall x \in \Z_n$
    
\end{itemize}


Public Key: $k = (b, n)$,
Private Key: $k\prime = (a, n)$

\begin{proof}
    Assume $\gcd(x, n) = 1$, then
    \[
        x^{\phi(n)} \equiv 1 \pmod n
    \]
    by Euler's Theorem.
    Since $ab \equiv 1 \pmod {\phi(n)}$,
    \[
        ab = \phi(n)m+1 \Rightarrow
        \begin{aligned}
            x^{ab} &= x^{\phi(n)m + 1}\\
            &= x {[x^{\phi(n)}]}^m\\
            &\equiv x \cdot 1^m \pmod n\\
            &=x
        \end{aligned}
    \]

    Suppose $\gcd(x, n) \neq 1$. Three cases
    \begin{itemize}
        \item $n\; |\; x \Rightarrow x^{ab} \equiv 0 \equiv x \pmod n$
        \item $p\; |\; x \text{ but } q \neq x$
        \[
            x^{ab} \equiv x \equiv 0 \pmod p
        \]
        On the other hand, by Fermat's little theorem
        \[
            x^{q-1 \equiv 1 \pmod q}
        \]
        It follows that
        \[
            \begin{aligned}
                x^{ab} &= x^{\phi(n)m+1}\\
                &= x^{(p-1)(q-1)m+1}\\
                &= x\cdot{[x^{q-1}]}^{(p-1)m}\\
                &\equiv x\cdot 1^{(p-1)n}\\
                &= x \pmod q
            \end{aligned}
        \]
    \end{itemize}
    
\end{proof}

\begin{proof}
    Why $k \not\Rightarrow k\prime$?

    If we know the prime factorization of 
    \[  
        n = pq
    \]
    then we can obtain $\phi(n) = (p-1)(q-1)$ easily.

    If we know $\phi(n)$, then a is the unique solution to the linear congruence
    \[
        bx \equiv 1 \pmod{\phi(n)}
    \]
    So the security of the RSA cryptosystem depends on the prime factorization of 
    $n = pq$
\end{proof}

\section{Primitive Roots \& Primes}

GOAL:\@ Find $n$ s.t. $n$ has a primitive root.

The aim of this section is to prove the existanceof primitive roots 
when $n = p$ is a prime

% \setcounter{theorem}{0}
\begin{theorem}
    Lagrange

    If $p$ is a prime and $f(x) = a_n x^n + \cdots + a_1x + a_0$
    is a polynomial with $a_0, a_1, \dots, a_n \in \Z$ and $p\;|\;a_n$, 
    then the congruence $f(x) \equiv 0 \pmod p$ has at most $n$ incongruent 
    solutions modulo $p$.
\end{theorem}
\begin{proof}
    The proof is by induction on $n$. If $n = 1$, then 
    \[
        f(x) = a_1 x + a_0
    \]
    and $f(x) \equiv 0 \pmod p \iff a_1 x \equiv -a_0 \pmod p$.
    Since $\displaystyle p \not \divides  a$, we have $\gcd(a_1, p) = 1$. Thus
    Theorem 4.7 implies that the linear congruence $a_1 x\equiv -a_0 \pmod p$
    has a unique solution modulo $p$.

    Suppose inductively that it holds true for all such polynomial with degree $n = k-1$
    and let $f(x) = a_n x^n + \cdots + a_1x + a_0$ with $a_0, a_1, \dots, a_n \in \Z$ and
    $p \not \divides  a_k$. We want to show $f(x) \equiv 0 \pmod p$ has at most $k$ congruent solutions
    modulo $p$.
    If $f(x) \equiv 0 \pmod p$ has no solution, then we are done.

    Suppose $f(x) \equiv 0 \pmod p$ has a solution, say $a \in \Z$, i.e. $f(a) \equiv 0 \pmod p$.
    Write
    \[
        f(x) = (x-a)q(x) + r
    \]
    for some $q(x) = b_{k-1} x^{k-1} + b_{k-2} x^{k-2} + \dots, b_1 x + b_0$
    with $b_0, b_1, \dots, b_{k-1} \in \Z$ and $b_{k-1} = a_k$

    Since $f(a) \equiv 0 \pmod p$, we find that 
    \[  
        r = f(a) \equiv 0 \pmod p
    \]
    Hence we have $f(x) \equiv (x-a)q(x) \pmod p$.

    If $h(x) = \sum_{j=0}^{n}{c_j x^j}$ \& $k(x) = \sum_{j=0}^{n}{d_j x^j}$
    are polynomials with integer coefficients, we write $h(x) \equiv k(x) \pmod p$
    if $c_j \equiv d_j \pmod p$ for $0 \leq j \leq n$

    Now let $b \in \Z$ be another solution of $f(x) \equiv 0 \pmod p$ with 
    $b \not\equiv a \pmod p$. We show that $q(b) \equiv 0 \pmod p$.

    Indeed, since $f(b) \equiv (b - a)q(b) \equiv 0 \pmod p, \Rightarrow p| (b-a)q(b)$.

    Since $b \not\equiv a \pmod p$, $\Rightarrow p \not \divides  (b-a) \Rightarrow p | q(b)$
    i.e. $q(b) \equiv 0 \pmod p$, so $b$ is a solution of $q(x) \equiv 0 \pmod p$.

    So we have shown that if $b \not\equiv a \pmod p$ is a solution of $f(x) \equiv 0 \pmod p$,
    thn $b$ is a solution of $q(x) \equiv \pmod p$.

    Since the degree of $q(x)$ is $k-1$ and its leading coefficient is not divisible by $p$, the induction hypothesis
    implies that $q(x) \equiv 0 \pmod p$ has at most 
    $k-1$ incongruent solutions modulo $p$ It follows that $f(x) \equiv 0 \pmod p$ has at most $k$ incongruent solutions modulo $p$
\end{proof}

Primitive root:
Let $a, n \in \Z$ with $n > 1$ and $\gcd(a, n) = 1$. The 
$a$ is a primitive root for n if $\text{ord}_n(a) = \phi(n)$

Goal of this section: 
Prove that if $n = p$ is a prime, then primitive roots for $p$ exists.

\begin{corollary}
    If $p$ is a prime \& $D|p-1$, then the congruence $x^d - 1 \equiv 0 \pmod p$ has exactly $d$
    incongruent solution modulo $p$.
\end{corollary}
\begin{proof}
    Let $p - 1 = dk$ for some $k \in \N$.
    Then
    \[
        x^{p-1}-1 = x^{dk}-1 = (x^d - 1)f(x)
    \]
    with $f(x) = x^{d(k-1)} + x^{d(k-2)} + \cdots + x^d + 1$.

    Since $f(x)$ has integer coefficient \& $\deg f(x) = d(k-1) = dk-d = p-1-d$

    Lagrange's Theorem implies that $f(x) \equiv 0 \pmod p$ has at most $p-1-d$ incongruent solution modulo $p$.
    on the other hand, by Fermat's little theorem, $x^{p-1} \equiv 0 \pmod p$ has $p-1$ incongruent solution modulo $p$, namely, $1, 2, \dots, p-1$

    Note that if $a$ is a solution of $x^{p-1}-1 \equiv 0 \pmod p$ but is \textbf{NOT} a solution of $f(x) \equiv 0 \pmod p$,
    then $a$ is a solution of $x^d-1 \equiv 0 \pmod p$. 
    Indeed, the assumption implies $0 \equiv a^{p-1} = (a^d - 1)f(a) \pmod p$

    
\end{proof}


\begin{eg}
    Let $p = 31 \Rightarrow \phi(p) = 3$.
    Take $d = 6$. Then Theorem 8.6 says that these are $\phi(6) = 2$ 
    incongruent integers having order $6$ modulo $31$.

    Let's find these two integers.

    Step 1 is to find a primitive root for $31$ (by trial and error).
    Since $2^5 = 32 \equiv 1 \pmod {31} \Rightarrow \text{ord}_{31}(2) = 5 \neq 31 \Rightarrow 2$ is Not a primitive root 
    for $31$.

    Next try $3$. It turns out that $3$ is a primitive root for $31$.

    \underbar{Why}? Let $d = \text{ord}_{31}(3)$. Then we know $d | 30 \Rightarrow d = 1, 2, 3, 5, 6, 10, 15, 30$.

    \underbar{Check}: $3^1 \equiv 3, 3^2 \equiv 9, 3^3 \equiv 27, 3^5 \equiv 26, 3^6 \equiv 16, 3^{10} \equiv 25, 3^{15} \equiv 30 \pmod {31}$
    $\Rightarrow d = 30 \Rightarrow 3$ is a primitive root for 31.
\end{eg}


\begin{remark}
    If $\text{ord}_{p}(a) = d$, then $\text{ord}_{p}(a^h) = \frac{d}{\gcd(h, d)}$
\end{remark}


\section{Composite integers having primitive roots}
\setcounter{theorem}{-1}
\begin{theorem}
    If $k \geq 3$, then $2^k$ has no primitive roots.
\end{theorem}
\setcounter{theorem}{7}
\begin{theorem}
    Let $m, n \in \N$ wiht $m, n > 2$ and $\gcd(m, n) = 1$.
    Then the integer $mn$ has no primitive roots
\end{theorem}
\begin{proof}
    Let $a \in \Z$ with $\gcd(a, mn) = 1$. Then $\gcd(a, m) = \gcd(a, n) = 1$.
    Put $h = \lcm(\phi(m), \phi(n))$ and $d = \gcd(\phi(m), \phi(n))$.
    Since $m > 2$ and $n > 2$, both $\phi(m)$ and $\phi(n)$ are even, hence $d >= 2$.

    It follows that
    \[
        h = \frac{\phi(m)\phi(n)}{d} \leq \frac{\phi(m)\phi(n)}{2} = \frac{\phi(mn)}{2} \Rightarrow h < \phi(mn)
    \]
    Next, we claim $a^h \equiv 1 \pmod mn$
    $\Rightarrow \ord_{mn}(a) \leq h < \phi(mn)$

    To prove the claim, we have
    \[
        a^{\phi(m)} \equiv 1 \pmod m
    \]
    by Euler's Theorem. Thus 
    \[
        \begin{aligned}
            a^h &\equiv {[a^{\phi(m)}]}^\frac{\phi(n)}{d}\\
            & \equiv 1^\frac{\phi(n)}{d}\\
            & \equiv 1 \pmod m
        \end{aligned}
    \]

    Similiar argument shows 
    \[
        a^h \equiv 1 \pmod n
    \]
    These give $m | a^h - 1$ and $n | a^h - 1$.

    Since $\gcd(m, n) = 1$, we get that
    \[
        mn \; | \; a^h - 1 \Rightarrow a^h \equiv 1 \pmod{mn}
    \]
    as desired.
\end{proof}

\begin{corollary}
    The integer $n$ fails to have a primitive root if either
    \begin{enumerate}
        \item $n$ is divisible by two \emph{odd} primes.
        \item $n$ is of the form $2^m p^k$ for some odd prime $p$ and positive
        integers $m, k$ with $m \geq 2$.
    \end{enumerate}
\end{corollary}
\begin{proof}
    Suppose $n$ is divisible by two odd primes, say $p \neq q$.

    Let $l \in \N$ s.t. $p^l | n$, but $p^{l+1} \not \divides  n$ and write $n=p^l h$ for 
    some $h \in \N$. 
    By assumption, $p \not \divides  h$ and $q | h$. 
    In particular, we see that $h \geq q > 2$.
    By Theorem 8.8, $n = p^l h$ has no primitive roots.

    Suppose $n = 2^m p^k$ for some \emph{odd} prime $p$ and $m, k \in \N$ with 
    $m \geq 2$.
    Then since $2^m > 2, p^k > 2$ and $\gcd(2^m, p^k) = 1$, $n$ has no 
    primitive roots by Theorem 8.8, again.
\end{proof}

\begin{remark}
    By Cor., if $n$ has a primitive root then $n$ is of the form $2, 4, p^k, 2p^k$
    where $p$ is an odd prime and $k \in \N$.
\end{remark}
\setcounter{lemma}{0}
\begin{lemma}
    If $p$ is an odd prime, then there exists a primitive root $r$ of $p$ s.t.
    \[
        r^{p-1} \not\equiv 1 \pmod {p^2}
    \]
\end{lemma}
\begin{proof}
    By Theorem 8.6, we know that $p$ has a primitive root $r$.
    If $r^{p-1} \not\equiv 1 \pmod {p^2}$, then we are done.
    If $r^{p-1} \equiv 1 \pmod {p^2}$, then we replace $r$ by
    $r\prime = r + p$. Note that $r\prime$ is a primitive root for $p$.
    We claim that 
    \[
        {(r\prime)}^{p-1} \not\equiv 1 \pmod {p^2}
    \]

    By the binomial theorem,
    \[
        \begin{aligned}
            {(r\prime)}^{p-1} &= {(r + p)}^{p-1} = \sum_{j=0}^{p-1} {p-1 \choose j} r^j p^{p-1-j}\\
            &\equiv r^{p-1} + {p-1 \choose p-2} r^{p-2} p \pmod {p^2}\\
            &\equiv r^{p-1} + (p-1) r^{p-2} p \pmod {p^2}\\
            &\equiv r^{p-1} + r^{p-2} p \pmod {p^2}\\
            &\equiv 1 + r^{p-2} p \pmod {p^2}
        \end{aligned}
    \]
    Since $p \not \divides  r$, we find that
    \[
        {(\prime r)}^{p-1} \equiv 1 - r^{p-2} p \not\equiv 1 \pmod {p^2}
    \]
    This proves the lemma.
\end{proof}

\begin{corollary}
    If $p$ is an odd prime, then $p^2$ has a primitive root.
    In fact, if $r$ is a primitive root of $p$, then $r$ or 
    $r + p$ is a primitive root of $p^2$.
\end{corollary}
\begin{proof}
    Note that $\phi(p^2) = p^2 - p = p(p-1)$.
    Let $r$ be a primitive root of $p$ with $r^{p-1} \not\equiv 1 \pmod {p^2}$ by lemma 1.

    Let $d = \ord_{p^2}(r)$. 
    We claim that $d = p(p-1)$.
    Indeed, since $r^{p-1} \not \equiv 1 \pmod {p^2}$, we see that $d \geq p-1$.
    On the other hand, as $d | \phi(p^2) = p(p-1)$, we must have $d = p(p-1)$.
    Therefore, $r$ is a primitive root of $p^2$.
\end{proof}

\begin{lemma}
    Let $p$ be an odd prime and $r$ be a primitive root of $p$ s.t. $r^{p-1} \not\equiv 1 \pmod {p^2}$.
    Then for each $k \geq 2$, we have $r^{p^{k-2}(p-1)} \not\equiv 1 \pmod{p^k}$.
\end{lemma}
\begin{proof}
    The proof is by induction on $k$. When $k = 2$, this follows from the assumption that
    $r^{p-1} \not\equiv 1 \pmod {p^2}$.
    Assume inductively that 
    \[  
        \tag{*}
        r^{p^{k-2}(p-1)} \not\equiv 1 \pmod {p^k}
        \label{eqn:8.3.1}
    \]

    We claim that $r^{p^{k-1}(p-1)} \not equiv 1 \pmod {p^{k+1}}$.
    Since $\gcd(r, p^{k-1}) = \gcd(r, p^k) = 1$,
    Eulers Theorem gives
    \[
        r^{\phi(p^{k-1})} = r^{p^{k-2}(p-1)} \equiv 1 \pmod p^{k-1}
    \]
    Hence there exists $a \in \Z$ s.t.
    \[
        r^{p^{k-2}(p-1)} = 1 + ap^{k-1}
    \]

    Then~\eqref{eqn:8.3.1} implies $p \not \divides  a$. By the binomial theorem, we find that
    \[
        \begin{aligned}
            r^{p^{k-1}(p-1)} &= {[r^{p^{k-2}(p-1)}]}^p \\
            &= {(1 + ap^{k-1})}^p \\
            &= \sum_{j=0}^{p}{p \choose j}{(ap^{k-1})}^j \\
            &\equiv 1 + p (ap^{k-1}) \\
            &\equiv 1 + ap^k \pmod p^{k+1}
        \end{aligned}
    \]
    Since $p \not \divides  a$, 
    \[
        r^{p^{k-1}(p-1)} \equiv 1 + ap^k \not\equiv 1 \pmod p^{k+1}
    \]
    This completes the proof.
\end{proof}

\begin{theorem}
    Let $p$ be an odd prime and $k \geq 1$ be a integer. Then 
    there exists a primitive root for $p^k$.
\end{theorem}
\begin{proof}
    By lemma 1 and lemma 2, there exists a primitive root $r$ of $p$ s.t. 
    $r^{p^{k-2}(p-1)} \not\equiv 1 \pmod {p^k}$.

    In fact, any $r$ with $r^{p-1} \not\equiv 1 \pmod {p^2}$ do the job.
    
    We claim that $r$ is a primitive root of $p^k$.

    Let $f = \ord_{p^k}(r)$. Then $d | \phi(p^k) = p^{k-1}(p-1)$.
    Since $r^d \equiv 1 \pmod {p^k}$, we have $r^d \equiv 1 \pmod p$.

    Therefore, $p - 1 | d$ as $r$ is a primitive root of $p$.
    it follows that $d = p^m(p-1)$ for some $0 \leq m \leq k-1$.
    If $m < k-1$, then $d | p^{k-2}(p-1)$, so that
    \[
        r^{p^{k-2}(p-1)} = {(r^d)}^{\frac{p^{k-2}(p-1)}{d}} \equiv 1^{\frac{p^{k-2}(p-1)}{d}} \equiv 1 \pmod {p^k}
    \]
    This is a contradiction, thus $m = k-1$, $d = p^{k-1}(p-1) = \phi(p^k)$.
    This shows that $r$ is a primitive root of $p^k$.
\end{proof}

\begin{corollary}
    Let $p$ be an odd prime and $k \in \N$. Then $2p^k$ has a primitive root.
\end{corollary}
\begin{proof}
    Let $r$ be a primitive root of $p^k$. We may assume
    $r$ is an odd integer.
    In fact, if $r$ is even, then we can replace $r$ by $r + p^k$, which is 
    an odd primitive root of $p^k$.

    We claim $r$ is also a primitive root of $2p^k$.
    Observe that $\gcd((r, 2p^k) = 1)$ and $\phi(2p^k) = \phi(2)\phi(p^k) = \phi(p^k)$.
    It follows that $d = \ord_{2p^k}(r) | \phi(p^k) = p^{k-1}(p-1)$.

    But $r^d \equiv \pmod {2p^k}$ implies, $r^d \equiv 1 \pmod {p^k}$;
    hence $\phi(p^k) = p^{k-1}(p-1) | d$, since $r$ is a primitive root of $p^k$.
    It follows that $d = p^{k-1}(p-1 = \phi(2p^k))$, therefore $r$ is a primitive root of $2p^k$.
\end{proof}

\section{Theory of Indecies}

Let $n > 1$ be a integer.
Suppose $n$ has a primitive root $r$.
Let $\{a_1, a_2, \dots, a_{\phi(n)}\}$ be the set of positive integers less than 
and relatively prime to $n$.

Then $\{r, r^2, \dots, r^{\phi(n)}\} \pmod{n} = \{a_1, a_2, \dots, a_{\phi(n)}\}$
Thus  for a given $a \in \{a_1, a_2, \dots, a_{\phi(n)}\}$ there exists
a unique $1 \leq k \leq \phi(n)$ s.t. $r^k \equiv a \pmod n$

\begin{eg}
    $n = 5, r = 2, \{a_1, a_2, \dots, a_{\phi(n)}\} = \{1, 2, 3, 4\}$

    $\Rightarrow \{2, 2^2, 2^3, 2^4\} \equiv \{2, 4, 3, 1\} \pmod n$

    $a = 3 \Rightarrow  k = 3$

    $a = 1 \Rightarrow  k = 4$
\end{eg}

\begin{definition}
    Let $r$ be a primitive root of $n$. If $\gcd(a, n) = 1$,
    then the smallest positive integer $k$ s.t. $a \equiv r^k \pmod n$ is 
    called the \emph{index of $a$ relative to $r$}, denoted by $k = \ind_r a$
\end{definition}

\begin{eg}
    $n=5, r=2$

    $\ind_2 3 = 3, \ind_2 1 = 4, \ind_2 2 = 1, \ind_2 4 = 2$
\end{eg}

\begin{remark}
    \hfill\break
    \begin{enumerate}
        \item The concept of index was introduced by Gauss.
        \item By definition, $1 \leq \ind_r a \leq \phi(n)$ and $r^{\ind_r a} \equiv a \pmod n$
        \item If $a \equiv b \pmod n$, then $\ind_r a = ind_r b$
    \end{enumerate}
\end{remark}

\begin{theorem}
    If $n$ has a primitive root $r$ and $a, b$ are integers with \newline
    $\gcd(a, n) = \gcd(b, n) = 1$, then the following properties hold.

    \begin{enumerate}
        \item $\ind_r(ab) \equiv \ind_r a + \ind_r b \pmod{\phi(n)}$
        \item $ind_r a^k \equiv k \ind_r a \pmod{\phi(n)}$ for $k > 0$
        \item $ind_r 1 \equiv 0 \pmod{\phi(n)}, ind_r r \equiv 1 \pmod{\phi(n)}$
    \end{enumerate}
\end{theorem}
\begin{proof}
    \begin{enumerate}
        \item By definition, $r^{\ind_r a} \equiv a \pmod n \& r^{\ind_r b} \equiv b \pmod n$,
        therefore
        \[
            \begin{aligned}
                ab &\equiv r^{\ind_r a} r^{\ind_r b} \\
                &\equiv r^{\ind_r a + \ind_r b} \pmod n
            \end{aligned}
        \]

        Since $r^{\ind_r ab} \equiv ab \pmod n$, we get $r^{\ind_r a + \ind_r b} \equiv r^{\ind_r ab} \pmod n$

        This implies $\ind_r a + \ind_r b \equiv \ind_r ab \pmod{\phi(n)}$

        \item We have 
        \[
            \begin{aligned}
                &r^{\ind_r 1} \equiv a \pmod n\\
                &\Rightarrow {(r^{\ind_r a})}^k \equiv a^k \pmod n\\
                &\Rightarrow r^{k\ind_r a} \equiv a^k \pmod n\\
            \end{aligned}
        \]
    \end{enumerate}
\end{proof}

\begin{theorem}
    If $n$ has a primitive root $r$ and $a, b \in \Z$
    with $\gcd(a, n) = \gcd(b, n) = 1$, then the following Properties hold:
    \begin{enumerate}
        \item $\ind_r ab \equiv \ind_r a + \ind_r b \pmod {\phi(n)}$
        \item $\ind_r a^k \equiv k \ind_r a \pmod {\phi(n)}$ for $k > 0$
        \item $\ind_r 1 \equiv 0 \pmod {\phi(n)}$, $\ind_r r \equiv 1 \pmod {\phi(n)}$
    \end{enumerate}
\end{theorem}

\begin{corollary}
    Suppose that $n$ has a primitive root $r$,
    Let $a, k$ be two integers with $k > 0$ and $\gcd(a, n) = 1$.
    Then the congruence equation $x^k \equiv a \pmod n$ has a solution if and only if 
    $\gcd(k, \phi(n)) | \ind_r a$.
\end{corollary}

\begin{corollary}
    The congruence equation
    \[
        x^m \equiv a \pmod n
    \]
    has a solution $\iff$ $\gcd(m, \phi(n)) | \ind_r a$
\end{corollary}
\begin{proof}
    \[
        \begin{aligned}
            \ind_r x^m &\equiv \ind_r a \pmod{\phi(n)}\\
            &\equiv m \ind_r x\\
            \Rightarrow my &\equiv \ind_r a \pmod{\phi(n)}
        \end{aligned}
    \]
\end{proof}

\begin{remark}
    The case $m = 2$ and $n = p$ is an odd prime is particularly interesting.
    In this case,
    \[
        \gcd(2, p-1) = 2
    \]
    Therefore, cor implies that $x^2 \equiv a \pmod p$ has a solution
    $\iff 2 \divides \ind_r a$ 
\end{remark}

\begin{theorem}
    Suppose that $n$ has a primitive root.
    Let $a, k$ be integers with $k > 0$ and $\gcd(a, n) = 1$
    Then  the congruence $x^K \equiv a \pmod n$ has a solution $\iff a^{\frac{\phi(n)}{d}} \equiv 1 \pmod n$
    where $d = \gcd(a, \phi(n))$.
    If it has a solution, then there are exactly $d$ incongruent solutions modulo $n$.
\end{theorem}
\begin{proof}
    We have shown that $x^K \equiv a \pmod n \iff d \divides \ind_r a$,
    where $r$ is a primitive root of $n$.
    Moreover, if it has a solution it has $d$ incongruent solutions modulo $n$.
    Thus to prove thm 8.12, it suffices to show $d \divides \ind_r a \iff a^{\frac{\phi(n)}{d}} \equiv 1 \pmod n$

    By taking indices and noting that $\ind_r 1 \equiv 0 \pmod {\phi(n)}$
    we get 
    \[
        \begin{aligned}
            a^{\frac{\phi(n)}{d}} \equiv 1 \pmod n &\iff \ind_r a^{\frac{\phi(n)}{d}} \equiv \ind_r 1 \equiv 0 \pmod {\phi(n)}\\
            &\iff \frac{\phi(n)}{d} \ind_r a \equiv 0 \pmod {\phi(n)}\\
            &\iff d \divides \inf_r a
        \end{aligned}
    \]
    This completes the proof.
\end{proof}

\begin{corollary}
    Let $p$ be a prime and $\gcd(a, p) = 1$.
    Then the congruence $x^k \equiv a \pmod p$ has a solution
    if and only if $a^{\frac{\phi(p)}{d}} \equiv d \pmod p$,
    where $d = \gcd(k, p-1)$
\end{corollary}
\begin{proof}
    By theorem 8.12 and the fact that $\phi(p) = p-1$
\end{proof}