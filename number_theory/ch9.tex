

\chapter{THE QUADRATIC RECIPROCITY LAW}

\section{Euler's Criterion}

\setcounter{theorem}{0}

Q. Let $p$ be and odd prime, $a, b, c \in \Z$ with $p \not \divides a$.
When will the quadratic congruence 
\[
    \label{eq:9.1.1}
    \tag{1}
    ax^2 + bx + c \equiv 0 \pmod p
\]
has a solution?

Let's try to simplify~\refeq{eq:9.1.1}. 
Since $p \not \divides a$ and $p$ is an odd prime, we have
$p \not \divides 4a$ and the equation~\refeq{eq:9.1.1} has a solution 
if and only if 
\[
    \label{eq:9.1.2}
    \tag{2}
    4a(ax^2 + bx + c) \equiv 0 \pmod p
\]
has a solution.
By using the identity
\[
    4a(ax^2 + bx + c) = {(2ax + b)}^2 - (b^2 - 4ac)
\]
The equation~\refeq{eq:9.1.2} becomes
\[
    \label{eq:9.1.3}
    \tag{3}
    {(2ax + b)}^2 \equiv (b^2 - 4ac) \pmod p
\]
Let $y = 2ax + b$ and $d = b^2 - 4ac$, equation~\refeq{eq:9.1.3} becomes 
\[
    \label{eq:9.1.4}
    \tag{4}
    y^2 \equiv d \pmod p
\]
Note that equation~\refeq{eq:9.1.1} is a solvable $\iff$ 
equation~\refeq{eq:9.1.4} is solvable


\bigbreak\hfill

\subsection{Conclusion:}

To solve quadratic congruences modulo an odd prime $p$, 
it suffices to know how to solve the quadratic congruences of the form
\[
    \label{eq:9.1.5}
    \tag{*}
    x^2 \equiv a \pmod p
\]
for $a \in \Z$


\begin{remark}
    \begin{enumerate}
        \item If $p \divides a$, then $x = p$ is an obvius solution of~\eqref{eq:9.1.5},
        so we may assume $p \not \divides a$ in the followings
        \item If~\eqref{eq:9.1.5} has a solution,
        then it has two incongruence solutions modulo $p$.
        Indeed, if $x_0$ is a solution of~\eqref{eq:9.1.5}, then $-x_0$ is also a solution.
        Moreover, $x_0 \not\equiv -x_0 \pmod p$, since otherwise we have $2x_0 \equiv 0 \pmod p$,
        i.e.
        \[
            \begin{aligned}
                p \divides 2x_0 \Rightarrow p \divides x_0\\
                \Rightarrow a \equiv x_0^2 \equiv 0 \pmod p
            \end{aligned}
        \]
    \end{enumerate}
\end{remark}


\section{The Legendre Symbol and It's Properties}

\begin{definition}
    Let $p$ be an odd prime and $a$ be an integer with $\gcd(a, p) = 1$, 
    the Legendre symbol
    \[
        (a/p) = \begin{cases}
            1  &\text{if } a \text{ is a quadratic residue of }p\\
            -1 &\text{if } a \text{ is a quadratic nonresidue of }p
        \end{cases}
    \]

    In other words, $(a/p) = 1$ if $x^2 \equiv a \pmod p$ has a solution,
    while $(a/p) = -1$ if $x^2 \equiv a \pmod p$ has no solution.

    We shall refer to $a$ as a numerator and $p$ as the denominator of the symbol $(a/p)$
\end{definition}

\begin{remark}
    Other standard notation for the Legendre symbol is
    \[
        (\frac{a}{p})
    \]
    and 
    \[
        (a \divides p)
    \]
\end{remark}

\begin{eg}
    Let $p = 13$, then the result of an easier example can be expressed as
    \[
        (1/13) = (3/13) = (4/13) = (9/13) = (10/13) = (12/13) = 1
    \]
    and
    \[
        (2/13) = (5/13) = (6/13) = (7/13) = (8/13) = (11/13) = -1
    \]
\end{eg}

\begin{remark}
    When $p \divides a$, the notation $(a/p)$ is undefined in this book. 
    However, in some books, $(a/p)$ is defined to be $0$ if $p \divides a$.
    One advantage of this is the number of solutions of $x^2 \equiv a \pmod p$
    can be given by the formula $1 + (a/p)$
\end{remark}

\begin{theorem}
    Let $p$ be an odd prime and $a, b$ be integers with $\gcd(a, p) = \gcd(b, p) = 1$.
    Then we have 
    \begin{enumerate}
        \item If $a \equiv b \pmod p$, then $(a/p) = (b/p)$
        \item $(a^2/p) = 1$
        \item $a^{\frac{p-1}{2}} \equiv (a/p) \pmod p$
        \item $(ab/p) = (a/p)(b/p)$
        \item $(1/p) = 1$ and $(-1/p) = {(-1)}^{\frac{p-1}{2}}$
    \end{enumerate}
\end{theorem}

\begin{remark}
    By (2) and (4) in Theorem 2, we see that
    \[
        (ab^2/p) = (a/p)(b^2/p) = (a/p)
    \]
    In other words, a square factor that is relatively prime to $p$ can be deleted
    from the numerator of Legendre symbol without affecting its value.
\end{remark}


\begin{corollary}
    If $p$ is an odd prime, then 
    \[
        (-1/p) = \begin{cases}
             1 &\text{if } p \equiv 1 \pmod 4\\
            -1 &\text{if } p \equiv 3 \pmod 4
        \end{cases}
    \]
\end{corollary}
\begin{proof}
    By (5) in Theorem 2, we have
    \[
        (-1/p) = {(-1)}^{\frac{p-1}{2}} = \begin{cases}
             1 &\text{if } p \equiv 1 \pmod 4 \\
            -1 &\text{if } p \equiv 3 \pmod 4
        \end{cases}
    \]
\end{proof}

\begin{theorem}
    9.3
    There are infinitely many primes of the form $4k + 1$
\end{theorem}
\begin{proof}
    Suppose there are finitely many primes of the form $4k + 1$, say $p_1, p_2, \dots, p_n$.
    Consider the integer $N = {(2p_1p_2 \cdots p_n)}^2 + 1$.
    Since $N$ is odd, $N$ is divisible by an odd prime $p$.
    This implies $N \equiv 0 \pmod p \Rightarrow {(2p_1p_2 \cdots p_n)}^2 \equiv -1 \pmod p$
    $\Rightarrow -1$ is a quadratic residue of $p$.
    
    By cor, $p \equiv 1 \pmod 4$, i.e. $p$ is of the form $4k + 1$, have $p = p_j$ for some $j = 1, 2, \dots, n$.
    But this gives 
    \[
        -1 \equiv {(2p_1p_2 \dots p_n)}^2 \equiv 0 \pmod p
    \]

    Thus there are infinitely many primes of the form $4k + 1$
\end{proof}


\begin{theorem}
    9.5 Gauss' lemma

    Let $p$ be an odd prime and $p \not\divides a$.
    If $n$ denotes the number of integers in the set $S = \{a, 2, \dots, (\frac{p-1}{2})a\}$,
    whose remainders upon division by $p$ exceed $\frac{p}{2}$, then $(a / p) = {(-1)}^n$
\end{theorem}

\begin{theorem}
    9.6

    if $p$ is an odd prime, then 
    \[
        (2/p) = \begin{cases}
            1  & \text{ if } p \equiv 1 \pmod 8 \text{ or } p \equiv 7 \pmod 8 \\
            -1 & \text{ if } p \equiv 3 \pmod 8 \text{ or } p \equiv 5 \pmod 8
        \end{cases}
    \]
\end{theorem}