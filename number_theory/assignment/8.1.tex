\documentclass[12pt]{exam}
\usepackage{amsthm}
\usepackage{libertine}
\usepackage[utf8]{inputenc}
\usepackage[margin=1in]{geometry}
\usepackage[fleqn]{amsmath}
\usepackage{amssymb}
\usepackage{cleveref}
\usepackage{mathtools}
\usepackage{multicol}
\usepackage[shortlabels]{enumitem}
\usepackage{siunitx}
\usepackage{cancel}
% \usepackage{graphicx}
\usepackage{pgfplots}
\usepackage{listings}
\usepackage{tikz}
% \usepackage[usenames,dvipsnames]{xcolor}


\pgfplotsset{width=10cm,compat=1.9}
\usepgfplotslibrary{external}
% \tikzexternalize

\newcommand{\class}{Number Theory} % This is the name of the course 
\newcommand{\examnum}{Homework 8.1} % This is the name of the assignment
\newcommand{\examdate}{17 Mar. 2023} % This is the due date
\newcommand{\timelimit}{}

\makeatother
\usepackage{thmtools}
\usepackage[framemethod=tikz]{mdframed}
\definecolor{NavyBlue}{RGB}{0, 0, 128}
\mdfsetup{skipabove=1em,skipbelow=0em}
\theoremstyle{definition}
\declaretheoremstyle[
    headfont=\bfseries\sffamily\color{NavyBlue!70!black}, bodyfont=\normalfont,
    mdframed={
        linewidth=2pt,
        rightline=false, topline=false, bottomline=false,
        linecolor=NavyBlue, backgroundcolor=NavyBlue!5,
    }
]{thmbluebox}
\declaretheorem[style=thmbluebox, numbered=no, name=Answer]{answer}

\begin{document}
\pagestyle{plain}
\thispagestyle{empty}

\noindent
\begin{tabular*}{\textwidth}{l @{\extracolsep{\fill}} r @{\extracolsep{6pt}} l}
      \textbf{\class} & \textbf{Name:} & \textit{Pohan Lin}\\ %Your name here instead, obviously 
      \textbf{\examnum} &&\\
      \textbf{\examdate} &&\\
\end{tabular*}\\
\rule[2ex]{\textwidth}{2pt}
% ---
8.1: 1, 2, 4, 5, 9, 11

\begin{enumerate}
      \item Find the order of the integers 2, 3, and 5:
            \begin{enumerate}
                  \item modulo 17.
                  \begin{answer}
                        Since $\phi(17) = 16$, all possible candidates are divisors 
                        of\newline $16 \Rightarrow \{1, 2, 4, 8, 16\}$.
                        \[
                              \begin{aligned}
                                    &2^2 \equiv 4 \pmod {17}\\
                                    &2^4 \equiv 16 \pmod {17}\\
                                    &2^8 \equiv 1 \pmod {17}\\
                              \end{aligned}
                        \]
                        \[
                              \begin{aligned}
                                    &3^2 \equiv 9 \pmod {17}\\
                                    &3^4 \equiv 13 \pmod {17}\\
                                    &3^8 \equiv 16 \pmod {17}\\
                                    &3^{16} \equiv 1 \pmod {17}\\
                              \end{aligned}
                        \]
                        \[
                              \begin{aligned}
                                    &5^2 \equiv 8 \pmod {17}\\
                                    &5^4 \equiv 13 \pmod {17}\\
                                    &5^8 \equiv 16 \pmod {17}\\
                                    &5^{16} \equiv 1 \pmod {17}\\
                              \end{aligned}
                        \]
                        $\text{ord}_{17}(2) = 8, \text{ord}_{17}(3) = 16, \text{ord}_{17}(5) = 16$
                  \end{answer}

                  \item modulo 19.
                  \begin{answer}
                        Since $\phi(19) = 18$, all possible candidates are divisors 
                        of\newline $18 \Rightarrow \{1, 2, 3, 6, 9, 18\}$.
                        \[
                              \begin{aligned}
                                    &2^2 \equiv 4 \pmod {19}\\
                                    &2^3 \equiv 8 \pmod {19}\\
                                    &2^6 \equiv 7 \pmod {19}\\
                                    &2^9 \equiv 18 \pmod {19}\\
                                    &2^{18} \equiv 1 \pmod {19}\\
                              \end{aligned}
                        \]
                        \[
                              \begin{aligned}
                                    &3^2 \equiv 9 \pmod {19}\\
                                    &3^3 \equiv 8 \pmod {19}\\
                                    &3^6 \equiv 7 \pmod {19}\\
                                    &3^9 \equiv 18 \pmod {19}\\
                                    &3^{18} \equiv 1 \pmod {19}\\
                              \end{aligned}
                        \]
                        \[
                              \begin{aligned}
                                    &5^2 \equiv 6 \pmod {19}\\
                                    &5^3 \equiv 11 \pmod {19}\\
                                    &5^6 \equiv 7 \pmod {19}\\
                                    &5^9 \equiv 1 \pmod {19}\\
                              \end{aligned}
                        \]
                        $\text{ord}_{19}(2) = 18, \text{ord}_{19}(3) = 18, \text{ord}_{19}(5) = 9$
                  \end{answer}

                  \item modulo 23.
                  \begin{answer}
                        Since $\phi(23) = 22$, all possible candidates are divisors 
                        of\newline $18 \Rightarrow \{1, 2, 11, 22\}$.
                        \[
                              \begin{aligned}
                                    &2^2 \equiv 4 \pmod {23}\\
                                    &2^{11} \equiv 1 \pmod {23}\\
                              \end{aligned}
                        \]
                        \[
                              \begin{aligned}
                                    &3^2 \equiv 9 \pmod {23}\\
                                    &3^{11} \equiv 1 \pmod {23}\\
                              \end{aligned}
                        \]
                        \[
                              \begin{aligned}
                                    &5^2 \equiv 6 \pmod {23}\\
                                    &5^{11} \equiv 22 \pmod {23}\\
                                    &5^{22} \equiv 1 \pmod {23}\\
                              \end{aligned}
                        \]
                        $\text{ord}_{23}(2) = 11, \text{ord}_{23}(3) = 11, \text{ord}_{23}(5) = 22$
                  \end{answer}
            \end{enumerate}
      \item Establish each of the statements below:
      \begin{enumerate}
            \item If $a$ has order $hk$ modulo $n$, then $a^h$ has order $k$ modulo $n$.
            \begin{answer}
                  By theorem 8.3, $a^h$ has order
                  \[
                        \frac{kh}{\gcd(h, kh)} = \frac{kh}{h} = k
                  \]
            \end{answer}

            \item If $a$ has order 2k modulo the odd prime $p$, then $ak \equiv -1 \pmod p$.
            \begin{answer}
                  \[
                        \begin{aligned}
                              a^{2k} &\equiv 1 \pmod p\\
                              a^{2k} - 1 &\equiv 0 \pmod 0\\
                              (a^k + 1)(a^k - 1) &\equiv 0 \pmod p\\
                              &\Rightarrow p\; |\; (a^k + 1)(a^k - 1)\\
                              &\Rightarrow p\; |\; a^k + 1 \lor p\; |\; a^k -1\\
                        \end{aligned}
                  \]
                  Assume that $p\; |\; a^k -1$ is true, then it contradict with $a^{2k} \equiv 1 \pmod p$,
                  thus we conclude that $p\; |\; a^k +1$ and $a^k \equiv -1 \pmod p$
            \end{answer}

            \item If $a$ has order $n - 1$ modulo $n$, then $n$ is a prime.
            \begin{answer}
                  The order of a number must divides $\phi(n)$.
                  Assume that $n$ is a composite and $n = ab$, then there can't be $n-1$ numbers relatively prime with n.
                  Therefore $\phi(n) < n - 1$, and the above is impossible. Thus conclude that $n$ must be prime.
            \end{answer}
      \end{enumerate}

      \setcounter{enumi}{3}
      \item Assume that the order of $a$ modulo $n$ is $h$ and the order of $b$ modulo $n$ is $k$. 
      Show that the order of $ab$ modulo $n$ divides $hk$; in particular, if $\gcd(h, k) = 1$, 
      then $ab$ has order $hk$.
      \begin{answer}
            
      \end{answer}

      \item Given that $a$ has order $3$ modulo $p$, where $p$ is an odd prime, show that $a + 1$
      must have order $6$ modulo $p$.\newline
      [Hint: From $a^2 + a + 1 \equiv 0 \pmod p$, it follows that ${(a + 1)}^2 \equiv a \pmod p$ and
      ${(a + 1)}^3 \equiv -1 \pmod p$.]
      \begin{answer}
      \end{answer}

      \setcounter{enumi}{8}
      \item \begin{enumerate}
            \item Verify that $2$ is a primitive root of $19$, but not of $17$.
            \begin{answer}
            \end{answer}

            \item Show that $15$ has no primitive root by calculating the orders of 
            $2, 4, 7, 8, 11, 13$, and $14$ modulo $15$.
            \begin{answer}
            \end{answer}
      \end{enumerate}

      \setcounter{enumi}{10}
      \item \begin{enumerate}
            \item Find two primitive roots of $10$.
            \begin{answer}
            \end{answer}
            
            \item Use the information that $3$ is a primitive root of $17$ 
            to obtain the eight primitive roots of $17$.
            \begin{answer}
            \end{answer}

      \end{enumerate}
\end{enumerate}

\end{document}
