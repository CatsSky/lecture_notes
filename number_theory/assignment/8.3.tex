\documentclass[12pt]{exam}
\usepackage{amsthm}
\usepackage{libertine}
\usepackage[utf8]{inputenc}
\usepackage[margin=1in]{geometry}
\usepackage[fleqn]{amsmath}
\usepackage{amssymb}
\usepackage{cleveref}
\usepackage{mathtools}
\usepackage{multicol}
\usepackage[shortlabels]{enumitem}
\usepackage{siunitx}
\usepackage{cancel}
% \usepackage{graphicx}
\usepackage{pgfplots}
\usepackage{listings}
\usepackage{tikz}
% \usepackage[usenames,dvipsnames]{xcolor}


\pgfplotsset{width=10cm,compat=1.9}
\usepgfplotslibrary{external}
% \tikzexternalize

\newcommand{\class}{Number Theory} % This is the name of the course 
\newcommand{\examnum}{Homework 8.3} % This is the name of the assignment
\newcommand{\examdate}{3 Apr. 2023} % This is the due date
\newcommand{\timelimit}{}

\makeatother
\usepackage{thmtools}
\usepackage[framemethod=tikz]{mdframed}
\definecolor{NavyBlue}{RGB}{0, 0, 128}
\mdfsetup{skipabove=1em,skipbelow=0em}
\theoremstyle{definition}
\declaretheoremstyle[
    headfont=\bfseries\sffamily\color{NavyBlue!70!black}, bodyfont=\normalfont,
    mdframed={
        linewidth=2pt,
        rightline=false, topline=false, bottomline=false,
        linecolor=NavyBlue, backgroundcolor=NavyBlue!5,
    }
]{thmbluebox}
\declaretheorem[style=thmbluebox, numbered=no, name=Answer]{answer}

\begin{document}
\pagestyle{plain}
\thispagestyle{empty}

\noindent
\begin{tabular*}{\textwidth}{l @{\extracolsep{\fill}} r @{\extracolsep{6pt}} l}
      \textbf{\class} & \textbf{Name:} & \textit{Pohan Lin}\\ %Your name here instead, obviously 
      \textbf{\examnum} &&\\
      \textbf{\examdate} &&\\
\end{tabular*}\\
\rule[2ex]{\textwidth}{2pt}
% ---
8.3: 1, 2, 3, 4, 5, 7

\begin{enumerate}
      \item \begin{enumerate}
            \item Find the four primitive roots of 26 and the eight primitive roots of 25.
            \begin{answer}

            \end{answer}
            \item Determine all the primitive roots of $3^2$, $3^3$, and $3^4$. 
            \begin{answer}

            \end{answer}
      \end{enumerate}

      \item For an odd prime p, establish the following facts:
      \begin{enumerate}
            \item There are as many primitive roots of $2p^n$ as of $p^n$.
            \begin{answer}
                  The number of primitive roots modulo $p^n$ is $\phi(\phi(p^n))$, and the number of primitive roots of $2p^n$ is $\phi(\phi(2p^n))$.
                  \[
                        \phi(\phi(2p^n)) = \phi(\phi(2)\phi(p^n))
                  \]
                  Since $\phi(2) = 1$
                  \[
                        \phi(\phi(2)\phi(p^n)) = \phi(\phi(p^n))
                  \]
                  Therefore the number of primitive roots modulo $p^n$ and $2p^n$ are the same.
            \end{answer}
            \item Any primitive root $r$ of $p^n$ is also a primitive root of $p$. \newline
            [Hint: Let $r$ have order $k$  modulo $p$. Show that $r^{pk} \equiv 1 \pmod {p^2}, \dots, r^{p^{n-1}k} \equiv 1 \pmod {p^n}$ and, hence, $\phi(p^n) | p^{n-1}k$.] 
            \begin{answer}
                  Let $k$ be tje order of $r$ modulo $p$.
                  \[
                        r^k \equiv 1 \pmod p \Leftrightarrow r^k = mp + 1
                  \]
                  for some integer $n$.

                  Then 
                  \[
                        r^{kp^j} \equiv {(mp + 1)}^{kp^j} \equiv {(1 + p^{j+1} m^j \dots)}^k \equiv 1 \pmod {p^{j+1}}
                  \]
                  For all integer $j \geq 1$, For $j = n$, we have
                  \[
                        r^{p^{n-1}k} \equiv 1 \pmod {p^n}
                  \]

                  Since $r$ is a primitive root, its order is $\phi(p^n)$ and so must divides $kp^{n-1}$
                  \[
                        \phi(p^n) = p^{n-1}(p-1) | p^{n-1}k
                  \]

                  Since $k$ is order of $r$ modulo $p$, $k \leq p-1$. Then we can conclude that it must be equal to $p - 1$,
                  and so the order of $r$ modulo $p$ is $\phi(p) = p-1$.
            \end{answer}
            \item A primitive root of $p^2$ is also a primitive root of $p^n$ for $n \geq 2$. 
            \begin{answer}
                  Let $r$ be a primitive root modulo $p^2$. $r$ has order $\phi(p^2) = p(p-1) > p-1$.
                  So
                  \[
                        r^{p-1} \not\equiv 1 \pmod {p^2}
                  \]

                  By the above part, we have that $r$ is a primitive root modulo $p$, such $r$ is also a primitive root modulo $p^n$
                  for any power of $p$ with $n > 1$
            \end{answer}
      \end{enumerate}

      \item If $r$ is a primitive root of $p^2$,  $p$ being an  odd prime, show that the  solutions of  the congruence
      $x^{p-1} \equiv 1 \pmod {p^2}$ are precisely the integers $r^p,  r^{2p},  \dots ,  r^{(p-1)p}$. 
      \begin{answer}
            Since $r$ is a primitive root of $p^2$, we consider the following
            \[
                  x^{p-1} \equiv 1 \pmod {p^2}
            \]

            Let $h$ be the order of $x$ satisfying this congruence. $h$ must divides $p-1$. Since $r$ is a primitive root, $x$
            can be expressed as $r^k$ for some $1 \leq k \leq p(p-1)$, the order of $r^k$ is 
            \[
                  h = \frac{\phi(p^2)}{\gcd(k, \phi(p^2))} = \frac{p(p-1)}{\gcd(k, p(p-1))}
            \]

            Since $h$ divides $p-1$, and $p$ doesn't divide $p-1$. We conclude that the denominator must have a factor of $p$.
            This happend iff $k$ is a multiple of $p$.

            We can conclude that any solution of the above euation must be in the form of $r^{mp}$ for some integer $1 \leq m \leq p-1$
      \end{answer}

      \item \begin{enumerate}
            \item Prove that $3$ is a primitive root of all integers of the form $7k$ and $2 \cdot 7^k$.
            \begin{answer}
                  $r = 3$ is a primitive root of $p = 7$.
                  \[
                        3^{p-1} \equiv 3^6 \equiv 43 \not\equiv 1 \pmod{7^2}
                  \]
                  This shows that $3$ is a primitive root of $7^2$ aswell, further more, this in fact implies that 
                  it is a primitive root of all powers of $7$.

                  A primitive root of $7^n$ is a primitive root of $2\cdot7^n$ aswell if it's an odd number. And $3$ is an odd
                  number, thus also a primitive root of $2\cdot7^n$
            \end{answer}
            \item Find a primitive root for any integer of the form $17^k$.
            \begin{answer}
                  A primitive root of $p = 17$ is given by $r = 3$
                  \[
                        3^{p-1} \equiv 3^{16} \equiv 171 \not\equiv 1 \pmod {17^2}
                  \]

                  This shows that $3$ is a primitive root of all powers of $17$.
            \end{answer}
      \end{enumerate}

      \item Obtain all the primitive roots of 41 and 82. 
      \begin{answer}
            For 41:

            We can find that 6 is a primitive root of 41. The primitive roots are given by the powers $6^k$ where $1 \leq k \leq \phi(41) = 40$ is
            relativly prime to 6. There are $\phi(40) = 16$ of them
            \[
                  3^{1}, 3^{3}, 3^{7}, 3^{9}, 3^{11}, 3^{13}, 3^{17}, 3^{19} \dots
            \]

            And they are congruent to 
            \[
                  6, 11, 29, 19, 28, 24, 26, 34, 35, 30, 12, 22, 13, 17, 15, 7
            \]


            Since $82 = 2 \cdot 41$, $\phi(82) = \phi(41) = 40$ primitive roots of 82 are congruent to the above list but have to be odd.
            Thus the primitive roots of 82 can be found by adding 41 to the even numbers of the list above.
            \[
                  47, 11, 29, 19, 69, 47, 75, 35, 71, 43, 53, 13, 17, 15, 7
            \]

      \end{answer}

      \setcounter{enumi}{6}
      \item Assume that $r$ is a primitive root of the odd prime $p$ and ${(r + tp)}^{p-1} \not\equiv 1 \pmod {p^2}$. 
      Show that $r + tp$ is a primitive root of $p^k$ for each $k \geq 1$. 
      \begin{answer}
            Let the order of $r + tp$ modulo $p^2$ be $h$
            \[
                  {(r+tp)}^h \equiv 1 \pmod {p^2} \Rightarrow {(r+tp)}^h \equiv 1 \pmod {p}
            \]

            The order of $r + tp \equiv r \pmod p$ is $p-1$ since $r$ is a primitive root. Therefore
            \[
                  p - 1 | h
            \]

            Write $h$ as $(p-1) k$ for some integer $1 \leq k \leq p$. The only way this is posible is if either $k = 1$ or $k = p$.
            Since $k = 1$ is excluded, we see that $k = p$ and $r + tp$ is a primitive root of $p^2$.

            Since $p^2$ is a primitive root of every higher powers, we can conclude that $r + tp$ is a primitive root of all $p^k$
      \end{answer}

\end{enumerate}

\end{document}
