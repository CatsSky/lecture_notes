\documentclass[12pt]{exam}
\usepackage{amsthm}
\usepackage{libertine}
\usepackage[utf8]{inputenc}
\usepackage[margin=1in]{geometry}
\usepackage[fleqn]{amsmath}
\usepackage{amssymb}
\usepackage{cleveref}
\usepackage{mathtools}
\usepackage{multicol}
\usepackage[shortlabels]{enumitem}
\usepackage{siunitx}
\usepackage{cancel}
% \usepackage{graphicx}
\usepackage{pgfplots}
\usepackage{listings}
\usepackage{tikz}
% \usepackage[usenames,dvipsnames]{xcolor}


\pgfplotsset{width=10cm,compat=1.9}
\usepgfplotslibrary{external}
% \tikzexternalize

\newcommand{\class}{Number Theory} % This is the name of the course 
\newcommand{\examnum}{Homework 8.4} % This is the name of the assignment
\newcommand{\examdate}{8 May. 2023} % This is the due date
\newcommand{\timelimit}{}

\makeatother
\usepackage{thmtools}
\usepackage[framemethod=tikz]{mdframed}
\definecolor{NavyBlue}{RGB}{0, 0, 128}
\mdfsetup{skipabove=1em,skipbelow=0em}
\theoremstyle{definition}
\declaretheoremstyle[
    headfont=\bfseries\sffamily\color{NavyBlue!70!black}, bodyfont=\normalfont,
    mdframed={
        linewidth=2pt,
        rightline=false, topline=false, bottomline=false,
        linecolor=NavyBlue, backgroundcolor=NavyBlue!5,
    }
]{thmbluebox}
\declaretheorem[style=thmbluebox, numbered=no, name=Answer]{answer}

\begin{document}
\pagestyle{plain}
\thispagestyle{empty}

\noindent
\begin{tabular*}{\textwidth}{l @{\extracolsep{\fill}} r @{\extracolsep{6pt}} l}
      \textbf{\class} & \textbf{Name:} & \textit{Pohan Lin}\\ %Your name here instead, obviously 
      \textbf{\examnum} &&\\
      \textbf{\examdate} &&\\
\end{tabular*}\\
\rule[2ex]{\textwidth}{2pt}
% ---
8.4: 1, 2, 3, 4, 5, 11, 12, 17

\begin{enumerate}

    \item Find the index of 5 relative to each of the primitive roots of 13.
    \begin{answer}
        
    \end{answer}

    \item Using a table of indices for a primitive root of 11, solve the following congruences:
    \begin{enumerate}
        \item $7x^3 \equiv 3 \pmod {11}$
        \item $3x^4 \equiv 5 \pmod {11}$
        \item $x^8 \equiv 10 \pmod {11}$
    \end{enumerate}

    

\end{enumerate}

\end{document}
