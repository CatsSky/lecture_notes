\documentclass[12pt]{exam}
\usepackage{amsthm}
\usepackage{libertine}
\usepackage[utf8]{inputenc}
\usepackage[margin=1in]{geometry}
\usepackage[fleqn]{amsmath}
\usepackage{amssymb}
\usepackage{cleveref}
\usepackage{mathtools}
\usepackage{multicol}
\usepackage[shortlabels]{enumitem}
\usepackage{siunitx}
\usepackage{cancel}
% \usepackage{graphicx}
\usepackage{pgfplots}
\usepackage{listings}
\usepackage{tikz}
% \usepackage[usenames,dvipsnames]{xcolor}


\pgfplotsset{width=10cm,compat=1.9}
\usepgfplotslibrary{external}
% \tikzexternalize

\newcommand{\class}{Number Theory} % This is the name of the course 
\newcommand{\examnum}{Homework 7.2} % This is the name of the assignment
\newcommand{\examdate}{17 Feb. 2023} % This is the due date
\newcommand{\timelimit}{}

\makeatother
\usepackage{thmtools}
\usepackage[framemethod=tikz]{mdframed}
\definecolor{NavyBlue}{RGB}{0, 0, 128}
\mdfsetup{skipabove=1em,skipbelow=0em}
\theoremstyle{definition}
\declaretheoremstyle[
    headfont=\bfseries\sffamily\color{NavyBlue!70!black}, bodyfont=\normalfont,
    mdframed={
        linewidth=2pt,
        rightline=false, topline=false, bottomline=false,
        linecolor=NavyBlue, backgroundcolor=NavyBlue!5,
    }
]{thmbluebox}
\declaretheorem[style=thmbluebox, numbered=no, name=Answer]{answer}


\begin{document}
\pagestyle{plain}
\thispagestyle{empty}

\noindent
\begin{tabular*}{\textwidth}{l @{\extracolsep{\fill}} r @{\extracolsep{6pt}} l}
    \textbf{\class} & \textbf{Name:} & \textit{Pohan Lin}\\ %Your name here instead, obviously 
    \textbf{\examnum} &&\\
    \textbf{\examdate} &&\\
\end{tabular*}\\
\rule[2ex]{\textwidth}{2pt}
% ---
7.2: 1, 2, 3, 5, 8, 11, 13, 16.

\begin{enumerate}
    \item Calculate $\phi(1001)$, $\phi(5040)$, and $\phi(36,000)$.
          \begin{answer}
              \[
                  \begin{aligned}
                      1001        & = 7 \cdot 11 \cdot 13                                                    \\
                      \phi(1001)  & = 1001 (1-\frac{1}{7}) (1-\frac{1}{11}) (1-\frac{1}{13})                 \\
                                  & = 720                                                                    \\
                      \\
                      5040        & = 2^4 \cdot 3^2 \cdot 5 \cdot 7                                          \\
                      \phi(5040)  & = 5040 (1-\frac{1}{2}) (1-\frac{1}{3}) (1-\frac{1}{5}) (1-\frac{1}{7}) s \\
                                  & = 1152                                                                   \\
                      \\
                      36000       & = 2^5 \cdot 3^2 \cdot 5^3                                                \\
                      \phi(36000) & = 36000 (1-\frac{1}{2}) (1-\frac{1}{3}) (1-\frac{1}{5})                  \\
                                  & = 9600                                                                   \\
                  \end{aligned}
              \]
          \end{answer}

    \item Verify that the equality $\phi(n) = \phi(n + 1) = \phi(n + 2)$ holds when $n = 5186$.

          \begin{answer}
              \begin{equation}
                  \tag{1}
                  \begin{aligned}
                      \phi(n) & = \phi(5186)                                               \\
                              & = \phi(2 \cdot 2593)                                       \\
                              & = 5186 \cdot (1 - \frac{1}{2}) (1 - \frac{1}{2593}) = 2592 \\
                  \end{aligned}
                  \label{eqn:2.1}
              \end{equation}

              \begin{equation}
                  \tag{2}
                  \begin{aligned}
                      \phi(n+1) & = \phi(5187)                                                                    \\
                                & = \phi(3 \cdot 7 \cdot 13 \cdot 19)                                             \\
                                & = 5187 (1-\frac{1}{3}) (1-\frac{1}{7}) (1-\frac{1}{13}) (1-\frac{1}{19}) = 2592 \\
                  \end{aligned}
                  \label{eqn:2.2}
              \end{equation}

              \begin{equation}
                  \tag{3}
                  \begin{aligned}
                      \phi(n+2) & = \phi(5188)                                     \\
                                & = \phi(2^2 \cdot 1297)                           \\
                                & = 5188 (1-\frac{1}{2}) (1-\frac{1}{1297}) = 2592
                  \end{aligned}
                  \label{eqn:2.3}
              \end{equation}

              From~(1),~(2),~(3), we can conclude that when $n = 5186$, $\phi(n) = \phi(n+1) = \phi(n+2)$ holds true.

          \end{answer}


    \item Show that the integers $m = 3^k \cdot 568$ and $n = 3^k \cdot 638$, where $k \geq 0$, satisfy
          simultaneously
          \[
              \tau(m) = \tau(n),\quad \sigma(m) = \sigma(n),\text{and} \quad \phi(m) = \phi(n)
          \]
          \begin{answer}
              \[
                  \begin{aligned}
                      m & = 3^k \cdot 2^3 \cdot 71        \\
                      n & = 3^k \cdot 2 \cdot 11 \cdot 29 \\
                  \end{aligned}
              \]

              \[
                  \begin{aligned}
                      \tau(m)   & = \tau(3^k)\tau(2^3)\tau(71)                           \\
                                & = \tau(3^k) \cdot 4 \cdot 2                            \\
                                & = \tau(3^k) \cdot 2 \cdot 2 \cdot 2                    \\
                                & = \tau(3^k)\tau(2)\tau(11)\tau(29) = \tau(n)           \\
                      \sigma(m) & = \sigma(3^k)\sigma(2^3)\sigma(71)                     \\
                                & = \sigma(3^k) \cdot 15 \cdot 72                        \\
                                & = \sigma(3^k) \cdot 2^3 \cdot 3^3 \cdot 5              \\
                                & = \sigma(3^k) \cdot 3 \cdot 12 \cdot 30                \\
                                & = \sigma(3^k)\sigma(2)\sigma(11)\sigma(29) = \sigma(n) \\
                      \phi(m)   & = \phi(3^k)\phi(2^3)\phi(71)                           \\
                                & = \phi(3^k) \cdot 4 \cdot 70                           \\
                                & = \phi(3^k) \cdot 2^3 \cdot 5 \cdot 7                  \\
                                & = \phi(3^k) \cdot 1 \cdot 10 \cdot 28                  \\
                                & = \phi(3^k)\phi(2)\phi(11)\phi(29) = \phi(n)           \\
                      % Induction
                      % &m_0 = 2^3 \cdot 71 \\
                      % &n_0 = 2 \cdot 11 \cdot 29 \\
                      % &\tau(m_0) = (3+1) \cdot (1+1) = 8 \\
                      % &\tau(n_0) = (1+1) \cdot (1+1) \cdot (1+1) = 8 \\
                      % &\sigma(m_0) = \sum{\{1, 2, 4, 8, 71, 142, 284, 568\}} = 1080 \\
                      % &\sigma(n_0) = \sum{\{1, 2, 11, 22, 29, 58, 319, 637\}} = 1080 \\
                      % ...
                  \end{aligned}
              \]
          \end{answer}

          \setcounter{enumi}{4}
    \item Prove that the equation $\phi(n) = \phi(n + 2)$ is satisfied by $n = 2(2 p - 1)$ whenever $p$ and
          $2 p - 1$ are both odd primes.
          \begin{answer}
              \[
                  \begin{aligned}
                      \phi(n + 2) & = \phi(2(2p-1)+2)             \\
                                  & = \phi(4p)                    \\
                                  & = \phi(4)\phi(p)              \\
                                  & = 2\cdot(p-1)                 \\
                                  & = 1\cdot(2p - 2)              \\
                                  & = \phi(2)\phi(2p-1) = \phi(n) \\
                  \end{aligned}
              \]
          \end{answer}

          \setcounter{enumi}{7}
    \item Prove that if the integer $n$ has $r$ distinct odd prime factors, then $2^r | \phi(n)$.
          \begin{answer}
              Since $n$ has $r$ distinct odd prime factors, we can write $n$ as followed
              \[
                  \begin{aligned}
                      n = 2^{k_0} p^{k_1}_1 \dots p^{k_r}_r
                  \end{aligned}
              \]
              where $k_i \in \mathbb{Z}$ for $i > 0$.
              \[
                  \begin{aligned}
                      \Rightarrow\phi(n) & = \phi(2^{k_0}) \phi(p^{k_1}_1) \dots \phi(p^{k_r}_r)                                 \\
                                         & = (\phi(2^{k_0}) \phi(p^{k_1-1}_1) \dots \phi(p^{k_r-1}_r)) \cdot (p_1-1)\dots(p_r-1)
                  \end{aligned}
              \]
              Each of the last $r$-terms is divisible by 2 since $p_1 \dots p_r$ are odd primes. \\
              Hence $2^r | \phi(n)$ holds true.
          \end{answer}

          \setcounter{enumi}{10}
    \item \begin{enumerate}
              \item If $\phi(n) | n - 1$, prove that $n$ is a square-free integer.\newline
                    [Hint: Assume that $n$ has the prime factorization $n = p_1^{k_1}p_2^{k_2} \dots p_r^{k_r}$,
                    where $k_1 \geq 2$.
                    Then $p_1 | \phi(n)$, whence $p_1 | n - 1$, which leads to a contradiction.]
                    \begin{answer}
                        Assume the contrary of the premise is true, that $n$ is not square-free.
                        Then $n = p^{k_1}_1 p^{k_2}_2 \dots p^{k_r}_r$ will have at least one prime factor with exponent greater than $1$.

                        Without loss of generality, let that be $p_1$. We then would have
                        \[
                            \begin{aligned}
                                \phi(n) & = \phi(p^{k_1}_1) \phi(p^{k_2}_2) \dots \phi(p^{k_r}_r)             \\
                                        & = p^{k_1-1}_1 (p_1 - 1) \cdot \phi(p^{k_2}_2) \dots \phi(p^{k_r}_r)
                            \end{aligned}
                        \]
                        The above shows that $p_1 | \phi(n) \Rightarrow p_1 | n-1$.\\
                        This contradict the fact that $p_1$ is a divisor of $n$ and $\gcd(n, n-1) = 1$.\\
                        Thus $n$ must be  a square-free integer.
                    \end{answer}

              \item Show that if $n = 2^k$ or $2^k 3^j$, with $k$ and $j$ positive integers, then $\phi(n) | n$.
                    \begin{answer}
                        Suppose that $n = 2^k$, then $\phi(n) = \phi(2^k) = 2^{k-1}$, which divides $n$.\\
                        Suppose that $n = 2^k3^j$ where $k, j \in \mathbb{Z}$, then
                        \[
                            \phi(n) = \phi(2^k)\phi(3^j) = 2^{k-1} 3^{j-1}
                        \]
                        which divides $n$.
                    \end{answer}

          \end{enumerate}

          \setcounter{enumi}{12}
    \item Assuming that $d | n$, prove that $\phi(d) | \phi(n)$.\newline
          [Hint: Work with the prime factorizations of $d$ and $n$.]
          \begin{answer}
              Let $n = p^{k_1}_1 p^{k_2}_2 \dots p^{k_r}_r$ be the prime factorization of $n$.
              Then we can represent every divisor of $n$ as $d = p^{j_1}_1 p^{j_2}_2 \dots p^{j_r}_r$, where $0 \geq j_i \geq k_i$ for $i = 1, 2, \dots, r$.

              From the above we now have
              \[
                  \phi(p^j) = p^{j-1}(p-1) \;|\; p^{k-1}(p-1) = \phi(p^k)
              \]
              Since the phi-function is multiplicative
              \[
                  \phi(d) = \phi(p^{j_1}_1) \phi(p^{j_2}_2) \dots \phi(p^{j_r}_r) \;|\; \phi(p^{k_1}_1) \phi(p^{k_2}_2) \dots \phi(p^{k_r}_r) = \phi(n)
              \]
          \end{answer}

          \setcounter{enumi}{15}
    \item Show that the Goldbach conjecture implies that for each even integer $2n$ there exist
          integers $n_1$ and $n_2$ with $\phi(n_1) + \phi(n_2) = 2n$.
          \begin{answer}
              Goldbach conjecture says that for each positive even integer greater
              than $2$ can be composed as the sum of two prime numbers.

              Let $n_1, n_2$ be two prime numbers s.t. $n_1 + n_2 = 2(n + 1)$, where $n \in \mathbb{Z}$.
              Then we have
              \[
                  \phi(n_1) + \phi(n_2) = (n_1 - 1) + (n_2 - 1) = 2n
              \]
              which solves the problem.
          \end{answer}

\end{enumerate}

\end{document}
