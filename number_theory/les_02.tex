\lesson{2}{di 14 Feb 2023 10:00}{}

Goal: Show that $\phi$ is multiplitive, e.g.
\begin{eg}
    \[
        \phi(m, n) = \phi(m) \phi(n)
        \text{if} \gcd(m, n) = 1
    \]
\end{eg}

\begin{lemma}
    Let $a, b, c$ be integers.
    Then $\gcd(a, b, c) = 1 \Leftrightarrow \gcd(a, b) = \gcd(b, c) = 1$
\end{lemma}
\begin{proof}
    $\Rightarrow$:
    Suppose there exists a prime $p$ s.t.
    \[
        p | \gcd(a, b)
    \]
    Then $p | a$ and $p | b$. But then $p | bc$ and hence $p | \gcd(a, bc) = 1$
    Thus we must have $\gcd(1, b) = 1$
    Similiar argument shows that $\gcd(a, c) = 1$

    $\Leftarrow$:
    Suppose that there exists a prime $p$ s.t. $p | \gcd(a, bc)$
    Then $p | a$ and $p | b$.
    Since $p$ is a prime, $p | b$ or $p | c$, say $p | b$
    But this implies $p | \gcd(a, b) = 1$, hence $p | c$
    But then $p | \gcd(a, c) = 1$
    Therefore, we conclude that $\gcd(a, bc) = 1$.
\end{proof}

\begin{theorem}
    7.2

    The function $\phi$ is multiplitive
\end{theorem}
\begin{proof}
    Let $m, n \in N$, with $\gcd(m, n) = 1$.
    Then we need to show $\phi(mn) = \phi(m) \phi(n)$.

    If $m = 1$ or $n = 1$, then since $\phi(1) = 1$,
    we have $\phi(mn) = \phi(m) \phi(n)$.
    So we may assume $n > 1$ and $m > 1$

    Let: arrange these integersas followed:

    \[
        \begin{matrix}
            1        & 2        & \cdots & r        & \cdots & m      \\
            m+1      & m+2      & \cdots & m+r      & \cdots & 2m     \\
            2m+1     & 2m+2     & \cdots & 2m+r     & \cdots & 3m     \\
            \vdots   & \vdots   &        & \vdots   &        & \vdots \\
            (n-1)m+1 & (n-1)m+2 & \cdots & (n-1)m+r & \cdots & nm
        \end{matrix}
    \]

    By definition, $\phi(mn) =$ the number of entries in this array that are relitively prime to $mn$.

    By lemma, this is the same as the number of integersthat are relitively prime to both $m, n$.

    Since $\gcd(qm+r, m) = \gcd(r, m)$, the numbers in the $r$-th column are relitively prime to $m \leftrightarrow \gcd(r, m) = 1$.

    Thus there are $\phi(m)$ columns containing integers prime to $m$.

    If we can show that for each of these columns, there are $\phi(n)$ integers taht are relitively prime to $n$, then we are done.

    Let $1 \leq r \leq m$ with $\gcd(r, m) = 1$.

    Then the entries of $r$-th column are
    \[
        r, m+r, 2m+r, \ldots, (n-1)m+r
    \]
    we claim that no two integers in the $r$-th column are conguence modulo $n$. Indeed, if
    \[
        hm + r \equiv km+r \pmod{n}
    \]
    for $0 \leq h, k \leq n-1$ then
    \[
        n | (hm+r) - (km+r) = (h-k)m
    \]
    Since $\gcd(n, m) = 1$, we find that
    \[
        n | h - k \Rightarrow h - k = 0, \text{ i.e. } h = k
    \]
    It follows that
    \[
        \{r, m+r, 2m+r, \ldots, (n-1)m+r\} \pmod{n}
        = \{0, 1, 2, \ldots, n-1\}
    \]
    Since $\gcd(qn+l, n) = \gcd(l, n)$, there are $\phi(n)$ integers in $r$-th column that are relitively prime to $n$.

    Therefore, there are total of $\phi(m)\phi(n)$ integers in this array that are relitively prime to both $m, n$.
\end{proof}

\begin{theorem}
    7.3

    Let $n > 1$ be an integer with the prime factorization.
    \begin{equation}   
        \tag{*} 
        n = \prod_{j=1}^{r}{p_j^{k_j}}
        \label{eqn:7.3.1}
    \end{equation}

    Then $\displaystyle \phi(n) = \prod_{j=1}^{r}{(p_j^{k_j} - p_j^{k_j-1})} = n \prod_{j=1}^{r}{(1-\frac{1}{p_j})}$.
\end{theorem}
\begin{proof}
    Since
    \[
        % \begin{array}
        \prod_{j=1}^{r}(p_j^{k_j} - p_j^{k_j-1}) = \prod_{j=1}^{r}{p_j^{k_j}(1-\frac{1}{p_j})}\\
        = \prod_{j=1}^{r}{p_j^{k_j}} \prod_{j=1}^{r}(1-\frac{1}{p_j})
        = n \prod_{j=1}^{r}(1-\frac{1}{p_j})
        % \end{array}
    \]
    it suffices to show that $\phi(n) = \prod_{j=1}^{r}{(p_j^{k_j} - p_j^{k_j-1})}$.

    For this, we use induction on $r$. If $r = 1$, then $n = p^k$ and we have $\phi(n) = \phi(p^k) = p^k - p^{k-1}$ by Theorem 7.1

    Suppose inductively that~\eqref{eqn:7.3.1} holds true for $r$.

    Let $n = \prod_{j=1}^{r}{p_j^{k_j}}$. Put $a = \prod_{j=1}^{r}{p_j^{k_j}}$ and $b = p_{r+1}^{k_r+1}$

    Then $n = ab$ and $\gcd(a, b = 1)$. By Theorem 7.2, Then 7.1, and the induction hypothesis, we get
    \[
        \phi(n) = \phi(ab) = \phi(a) \phi(b)
        = \prod_{j=1}^{r}{(p_j^{k_j} - p_j^{k_j-1})} (p_{r+1}^{k_{r+1}})
        % need to check again
        = \prod_{j=1}^{r+1}(p_j^{k_j} - p_j^{k_j-1})
    \]
    This completes the proof.
\end{proof}

\begin{theorem}
    7.4

    For $n > 2$, $\phi(n)$ is an even integer.
\end{theorem}
\begin{proof}
    If $n$ has an odd prime divisor $p$, then $n = p^k m$ for some $k, m \in \mathbb{N}$
    with $\gcd(p, m) = 1$.

    By theorem 7.1 and theorem 7.2.
    \[
        \phi(n) = \phi(p^k m) = \phi(p^k) \phi(m) = (p^k - p^{k - 1} \phi(m))
    \]

    Since $p^k - p^{k-1}$ is an even integer, $\phi(n)$ is an even integer

    If there is no such $p$, then $n = 2^k$ for some $k \in \mathbb{N}$.
    Since $n > 2$, we have $k \geq 2$. By theorem 7.1,
    \[
        \phi(n) = \phi(2^k) = 2^k - 2^{k-1} = 2^{k-1}
    \]
    is an even integer
\end{proof}

\begin{corollary}
    There are $\inf$-many primes.
\end{corollary}
\begin{proof}
    The proof is by contradiction. Suppose these finitely many primes, say $p_1, p_2, \ldots, p_n$.
    Let
    \[
        n = \prod_{j=1}^{r}p_j
    \]

    We can claim that, if $1 < a \leq n$, then $\gcd(a, n) \neq 1$.

    By the fundamental theorem od arithmetics, there exists prime $q$ s.t. $q | a$.
    By our assumption, $q$ is one of $p_1, p_2, \dots, p_r$, say $q = p_1$.
    Since $n = \prod_{j=1}^{r}p_j$, we conclude that $q | \gcd(a, n) \Rightarrow \gcd(a, n) \neq 1$.

    Thus there is only one integer between $1$ and $n$ that is relitively prime, namely $1 \Rightarrow \phi(n) = 1 \Rightarrow n = 1, 2$
\end{proof}

\setcounter{section}{2}
\section{Euler's Theorem}

\begin{theorem}
    7.5

    Let $a, n$ be integers with $n \geq 1$.

    If $\gcd(a, n) = 1$, then $a^\phi(n) \equiv 1 \pmod n$

\end{theorem}
\begin{remark}
    If $n = p$ is a prime, then $\phi(n) = \phi(p) = p - 1$, and hence theorem 7.5 becomes $a^{p-1} \equiv 1 \pmod n$
    which is the content of the Fermat's little theorem.

    Therefore, Euler's theorem is a generalization of Fermat's little theorem.
\end{remark}

\subsection*{Goal: Prove Euler's Theorem}
\begin{lemma}
    Let $a, n$ be integers with $n > 1$ and $\gcd(a, n) = 1$.

    If $a_1, a_2, \dots, a_{\phi(n)}$ are the positive integers less than $n$ and relatively prime to $n$, then
    $aa_1, aa_2, \dots, aa_{\phi(n)}$ are conguent modulo $n$ to \\ $a_1, a_2, \dots, a_{\phi(n)}$ in same order

    i.e.
    \[
        \tag*{*}
        \{aa_1, aa_2, \dots, aa_{\phi(n)}\} \pmod n = \{a_1, a_2, \dots, a_{\phi(n)}\}
        \label{eqn:7.5.1}
    \]
\end{lemma}
\begin{proof}
    Let $1 \leq i \leq \phi(n)$. Let $1 \leq b_i < n$ be the remainder of $aa_i$
    divided by $n$.

    We want to show that $\exists! \; 1 \leq j \leq \phi(n)$ s.t. $b_i = a_j$.

    For this, we prove~\eqref{eqn:7.5.1}. We first show that all $b_i$ are distinct, Suppose there exist
    $1 \leq i, j \leq \phi(n)$ s.t. $b_i = b_j$.

    By definition, we have $aa_i \equiv b_i \equiv b_j \equiv aa_j \pmod n$.

    This implies $n | aa_i -aa_j = a(a_i - a_j)$.
    Since $\gcd(a, n) = 1$, we have $n | a_i - a_j$.
    Since $1 \leq a_i, a_j \leq n$, $|a_i - a_j| \leq n - 1$.

    Therefore $n|a_i - a_j \Rightarrow a_i = a_j$, i.e. $i = j$.

    To complete the proof, we check that $\gcd(b_i, n) = 1$ for $1 \leq i \leq \phi(n)$.
    Suppose there exist $1 \leq i \leq \phi(n)$ and $a \; \text{prime}\; p$ a.t.
    $p | \gcd(b_i, n)$. By definition, there exists $q \in \Z$ s.t.
    $aa_i = nq + b_i$. 
    
    Since $p|\gcd()b_i, n$, we have $p | b_i$ and $p | n$. 
    This implies $p | aa_i$ and hence $p | a$ or $p | a_i$.
    
    If $p | a$, then since $p | n$, we get $p | \gcd(n, a) = 1$.

    If $p | a_i$, then since $p | n$, we find $p | \gcd(n, a_i) = 1$.

    Therefore we must have $\gcd(b_i, n) = 1$ for all $1 \leq i \leq \phi(n)$
\end{proof}
\begin{eg}
    Let $n = 9$. 
    
    Then positive integers less than are relatively prime to $9$ are $1, 2, 4, 5, 7, 8$
    These integers play the role of the integers $a_1, a_2, \dots, a_{\phi(n)}$ in the lemma.

    If $a = -4$, then the integers $aa_i$ are $-4, -8, -16, -20, -28, -32$.

    Where modulo $9$, 
    \[  
        \{-4, -8, -16, -20, -28, -32\} \equiv \{5, 1, 2, 7, 8, 4\} \pmod 9
    \]

    Therefore,
    \[
        \begin{aligned}
            &(-4)\cdot(-8)\cdot(-16)\cdot(-20)\cdot(-28)\cdot (-32) \equiv 1\cdot2\cdot4\cdot5\cdot7\cdot8 \pmod 9 \\
            &\Rightarrow (-4\cdot1)\cdot(-4\cdot2)\cdot(-4\cdot4)\cdot(-4\cdot7)\cdot(-4\cdot8) = \\ 
            &(-4)^6\cdot(1\cdot2\cdot4\cdot5\cdot7\cdot8) \equiv 1\cdot2\cdot4\cdot5\cdot7\cdot8 \pmod 9 \\
            &\Rightarrow 9 | [(-4)^6 - 1]\cdot (1\cdot2\cdot4\cdot5\cdot7\cdot8) \\
            &\Rightarrow 9 | (-4)^6 - 1 \; \text{,i.e.} \; (-4)^6 \equiv 1 \pmod 9 \\
        \end{aligned}
    \]
\end{eg}

\begin{theorem} 7.5

    Let $a, n \in \N$ with $\gcd(a, n) = 1$.
    Then
    \[
        a^{\phi(n)} \equiv 1 \pmod n
    \]
\end{theorem}
\begin{proof} 7.5
    
    We may assume that $n > 1$ as Theorem 7.5 holds trivially when $n = 1$.

    Let $a_1, \dots, a_{\phi(n)}$ be positive integers less than and relitively prime to $n$.

    Since $\gcd(a, n = 1$), lemma implies $aa_1, \dots, aa_{\phi(n)}$ are conguent to
    $a_1, \dots, a_{\phi(n)}$ modulo $n$, in same order.

    Therefore, we have
    \[
        \begin{aligned}
            &aa_1 \cdot aa_2 \dots aa_{\phi(n)} \equiv a_1a_2\dots a_{phi(n)} \pmod n \\
            &\Rightarrow a^{\phi(n)}(a_1a_2\dots a_{\phi(n)}) \equiv a_1a_2\dots a_{\phi(n)} \pmod n \\
            &\Rightarrow n \;|\; (a^{\phi(n)} - 1)\cdot(a_1a_2\dots a_{\phi(n)})
        \end{aligned}
    \]

    Since $\gcd(n, a_i) = 1$ for $1 \leq i \leq \phi(n)$, we have $\gcd(n, a_1a_2\dots a_{\phi(n)}) = 1$.

    And hence $n | a^{\phi(n)} - 1$, i.e. $a^{\phi(n)} \equiv 1 \pmod n$.


\end{proof}
